\documentclass[aspectratio=169]{beamer}
% \usepackage[utf8]{inputenc}
\usepackage[T1]{fontenc}
\usetheme{metropolis}
\usecolortheme{orchid}
\usepackage{amsmath}
\usepackage{amssymb}
\usepackage{amsthm}
\usepackage{multirow}
\usepackage[ruled]{algorithm2e}
\usepackage{mathtools}
\usepackage{caption}
% \usepackage{epstopdf}
\usepackage{hyperref}
\usepackage{subcaption}
\usepackage{siunitx}
\usepackage{textcomp}
\usepackage{tcolorbox}
\usepackage[framed,numbered]{matlab-prettifier}

\newfontfamily{\lstubuntumono}{Ubuntu Mono}
\lstset{
  style              = Matlab-editor,
  basicstyle				 =\lstubuntumono\tiny,
  escapechar         = ",
  mlshowsectionrules = false,
}
 
\usepackage{tikz}
\usetikzlibrary{tikzmark,positioning,arrows.meta,shapes.misc,calc}

\setbeamerfont{footnote}{size=\tiny}

% Information boxes
\newcommand*{\info}[4][16.3]{%
  \node [ annotation, #3, scale=0.65, text width = #1em,
          inner sep = 2mm ] at (#2) {%
  \list{$\bullet$}{\topsep=0pt\itemsep=0pt\parsep=0pt
    \parskip=0pt\labelwidth=8pt\leftmargin=8pt
    \itemindent=0pt\labelsep=2pt}%
    #4
  \endlist
  };
}

%% Definitions for root locus plots
\newcommand*{\rootlocusexample}[4]{% xmin,xmax,ymin,ymax,
		\foreach \x in {#1,...,#2}{
			\ifthenelse{\x=0}{}
		  {% false case
   			\draw (\x cm,1pt) -- (\x cm,-1pt) node[anchor=north] {$\x$};
		  }
		}
		\foreach \y in {#3,...,#4}{
			\ifthenelse{\y=0}{}
		  {% false case
   			\draw (-1pt,\y cm) -- (1pt,\y cm) node[anchor=east] {$\y$};
		  }
   	}
		\draw [-latex] (#1,0) -- (#2,0) node [above]  {$\sigma$};
		\draw [-latex] (0,#3) -- (0,#4) node [right] {$j\omega$};
}

\def\centerarc[#1](#2)(#3:#4:#5)% Syntax: [draw options] (center) (initial angle:final angle:radius)
    { \draw[#1] ($(#2)+({#5*cos(#3)},{#5*sin(#3)})$) arc (#3:#4:#5); }

\tikzset{%
  >={Latex[width=2mm,length=2mm]},
  % Specifications for style of nodes:
            base/.style = {rectangle, rounded corners, draw=black,
                           minimum width=4cm, minimum height=1cm,
                           text centered, font=\sffamily},
  activityStarts/.style = {base, fill=blue!30},
       startstop/.style = {base, fill=red!30},
    activityRuns/.style = {base, fill=green!30},
         process/.style = {base, minimum width=2.5cm, fill=orange!15,
                           font=\ttfamily},
}

\tikzset{pole/.style={cross out, draw=black, minimum size=2*(#1-\pgflinewidth), inner sep=0pt, outer sep=0pt},
pole/.default={3pt}}
\tikzset{poledes/.style={rectangle, draw=black, minimum size=2*(#1-\pgflinewidth), inner sep=0pt, outer sep=0pt},
poledes/.default={3pt}}
\tikzset{zero/.style={circle, draw=black, fill=white, minimum size=2*(#1-\pgflinewidth), inner sep=0pt, outer sep=0pt},
zero/.default={3pt}}
\tikzset{test/.style={rectangle, draw=black, fill=white, minimum size=2*(#1-\pgflinewidth), inner sep=0pt, outer sep=0pt},
test/.default={3pt}}

%% Definitions for block diagrams
\tikzstyle{block} = [draw, fill=blue!20, rectangle, 
    minimum height=2em, minimum width=3em]
\tikzstyle{sum} = [draw, fill=blue!20, circle, node distance=1cm]
\tikzstyle{input} = [coordinate]
\tikzstyle{output} = [coordinate]
\tikzstyle{pinstyle} = [pin edge={to-,thin,black}]

\renewcommand\textbullet{\ensuremath{\bullet}}
\newcommand\scalemath[2]{\scalebox{#1}{\mbox{\ensuremath{\displaystyle #2}}}}
\newcommand{\norm}[1]{\left\lVert#1\right\rVert}

%%% Bibliography
\usepackage[citestyle=numeric,style=numeric,backend=biber,doi=false,isbn=false,url=false]{biblatex}
\addbibresource{references.bib}

%%% Suppress biblatex annoying warning
\usepackage{silence}
\WarningFilter{biblatex}{Patching footnotes failed}

%%% new theorems %%%%%%%%%%%%%%%%%%%%%%%%%%%%%%%%%%%%%%%%%%%%%%%%%%%%%%%%%%%%%%
\theoremstyle{definition}
\newtheorem{mydef}{Definition}

\theoremstyle{plain}
\newtheorem{mylemma}{Lemma}[section]
\newtheorem{mytheorem}{Theorem}[section]
\newtheorem{myproposition}{Proposition}[section]
\newtheorem{myproblem}{Problem}[section]
\newtheorem{mydefinition}{Definition}[section]
\newtheorem{myassumption}{Assumption}[section]

\theoremstyle{remark}
\newtheorem{myremark}{Remark}[section]

\newcounter{saveenumi}
\newcommand{\seti}{\setcounter{saveenumi}{\value{enumi}}}
\newcommand{\conti}{\setcounter{enumi}{\value{saveenumi}}}

\resetcounteronoverlays{saveenumi}

\title{Control de Sistemas}
\subtitle{\small Clase 10: Introducción al Control Digital}
\author{Gerardo Becerra, Ph.D.}
\institute{Pontificia Universidad Javeriana\\ Departamento de Electrónica}
\date{Abril 24, 2020}

\begin{document}

\frame{\titlepage}	

\begin{frame}[<+->]\frametitle{Introducción}
\centering
\begin{itemize}
	\item Control analógico estudiado hasta el momento $\rightarrow$ se implementa usando circuitos electrónicos analógicos.
	\item Actualmente, los sistemas de control se implementan usando sistemas digitales (microprocesadores, microcontroladores).
	\item Su uso se ha incrementado a medida que su precio disminuye y sus capacidades aumentan.
	\item Sistema de control digital $\rightarrow$ usa señales digitales y un computador digital para controlar el proceso.
	\item Se requieren circuitos especiales para convertir las señales entre sus representaciones digital / analógica y actuar como interfaz entre el controlador y el proceso.
\end{itemize}
\end{frame}

\begin{frame}[<+->]\frametitle{Elementos de un Sistema de Control Digital}
	\begin{figure}[h]
		\centering
		\includegraphics[width=9cm]{images/digitalControlSystem.eps}
	\end{figure}
	\small
	\vspace*{-2mm}
	\begin{columns}
		\begin{column}{0.5\textwidth}
			\begin{itemize}
				\item Actuador, proceso: sistemas físicos que poseen señales analógicas.
				\item Sensor: captura la señal de salida de la planta, generando una señal eléctrica.
				\item Convertidor analógico-digital (ADC): transforma la señal eléctrica generada por el sensor a una representación digital.
			\end{itemize}
		\end{column}
		\begin{column}{0.5\textwidth}
			\begin{itemize}
				\item Computador digital: implementa la función del controlador/compensador. Calcula la ley de control a partir de la referencia y la medición y genera una señal de control digital.
				\item Convertidor digital-analógico (DAC): transforma la señal de control generada por el computador en una representación analógica.
			\end{itemize}
		\end{column}
	\end{columns}
\end{frame}

\begin{frame}[<+->]\frametitle{Elementos de un Sistema de Control Digital}
	\small
	\begin{itemize}
		\item Controladores analógicos $\rightarrow$ se construyen usando circuitos analógicos (resistores, capacitores, amplificadores operacionales).
		\item Dichos circuitos realizan implícitamente la operación de integración.
		\item Integración: operación fundamental para la solución de ecuaciones diferenciales.
		\item Computadores digitales $\rightarrow$ no pueden integrar.
		\item Integración numérica: aproximación de la solución usando una ecuación algebráica.
		\item Diseño de controladores digitales:
		\begin{itemize}
			\item Diseño del controlador en tiempo continuo y discretización del controlador obtenido $\rightarrow$ \textcolor{green}{implementación directa usando los métodos que ya conocemos,} \textcolor{red}{puede presentar errores de aproximación.}
			\item Diseño del controlador directamente en el dominio digital $\rightarrow$ \textcolor{green}{no hay errores de aproximación,} \textcolor{red}{se requiere conocer métodos específicos de sintonización.}
		\end{itemize}
	\end{itemize}
\end{frame}

\begin{frame}[<+->]\frametitle{Elementos de un Sistema de Control Digital}
	% \vspace*{5mm}
	\begin{columns}
		\begin{column}{0.5\textwidth}
			\begin{figure}
				\centering
				\includegraphics[width=7.5cm]{images/digitalControlSystem2.eps}
			\end{figure}
			Convertidor analógico - digital (ADC):
			\begin{itemize}
				\item Convierte el voltaje $y(t)$ en un número binario (10 - 12 bits).
				\item 10 bits: $2^{10} = 1024$ valores $\rightarrow$ resolución de $0.1 \%$.
			\end{itemize}
		\end{column}
		\begin{column}{0.5\textwidth}
			\begin{itemize}
				\item La conversión de $y(t)$ se realiza con un periodo de muestreo $T$ obteniendo la señal discreta $y(kT)$, $k \in \mathbb{Z}$.
			\end{itemize}
			\pause
			Referencia $r(kT)$:
			\begin{itemize}
				\item Usando un sampler-ADC para digitalizar una señal analógica $r(t)$.
				\item Valor digital obtenido desde un puerto de comunicaciones.
			\end{itemize}
		\end{column}
	\end{columns}
\end{frame}

\begin{frame}[<+->]\frametitle{Elementos de un Sistema de Control Digital}
	\vspace*{-3mm}
	\begin{columns}
		\begin{column}{0.5\textwidth}
			\begin{figure}
				\centering
				\includegraphics[width=7.5cm]{images/digitalControlSystem2.eps}
			\end{figure}
		\end{column}
		\begin{column}{0.5\textwidth}
			Convertidor digital - analógico:
			\begin{itemize}
				\item Convierte el número binario $u(kT)$ en un voltaje $u(t)$.
				\item Un retenedor de orden cero (ZOH) mantiene el voltaje durante el periodo de muestreo.
				\item El voltaje $u(t)$ se aplica sobre el actuador como en el caso de un sistema de control analógico.
			\end{itemize}
		\end{column}
	\end{columns}
\end{frame}

\begin{frame}[<+->]\frametitle{Elementos de un Sistema de Control Digital}
	\vspace*{-2mm}
	\begin{columns}
		\begin{column}{0.5\textwidth}
			\begin{figure}
				\centering
				\includegraphics[width=7.5cm]{images/digitalControlSystem2.eps}
			\end{figure}
			Controlador digital:
			\begin{itemize}
				\item Opera sobre muestras de la señal de salida de la planta.
				\item Dinámica del controlador $\rightarrow$ se implementa usando ecuaciones de diferencias.
			\end{itemize}
		\end{column}
		\begin{column}{0.5\textwidth}
			\begin{itemize}
				\item Diseño:
				\begin{itemize}
					\item Discretización de un controlador de tiempo continuo.
					\item Métodos de sintonización de controladores discretos.
				\end{itemize}
				\item Métodos de discretización:
				\begin{itemize}
					\item Backward
					\item Tustin
					\item Mapeo de polos y ceros
				\end{itemize}
				\item Se requiere un $T$ suficientemente pequeño para obtener una buena aproximación.
			\end{itemize}
		\end{column}
	\end{columns}
\end{frame}

\begin{frame}[<+->]\frametitle{Sistemas de Datos Muestreados}
	\vspace*{3mm}
	\begin{itemize}
		\item Señal discreta: Señal obtenida a partir de una señal de tiempo contínuo $x(t)$ al tomar muestras en instantes de tiempo discretos $x(kT)$.
		\item Sistema de datos muestreados: Sistema donde parte de éste tiene señales discretas.
		\item Muestreador (sampler): switch que se cierra por un instante cada $T$ segundos.
	\end{itemize}
	\begin{figure}
		\centering
		\includegraphics[width=13cm]{images/sampler.eps}
	\end{figure}
\end{frame}

\begin{frame}[c]\frametitle{Sistemas de Datos Muestreados}
	\vspace*{-2mm}
	\begin{itemize}
		\item Muestreador-retenedor (sample and hold): sistema que mantiene el valor de la muestra tomada durante el periodo de muestreo $T$.
	\end{itemize}
	\begin{figure}
		\centering
		\includegraphics[width=13cm]{images/sampleHold.eps}
	\end{figure}
	\pause
	\vspace*{-5mm}
	\begin{columns}
		\begin{column}{0.6\textwidth}
			\begin{itemize}
				\item Retenedor de orden cero (zero order hold):
			\end{itemize}
				\vspace*{-3mm}
				\begin{align*}
					p(t) &= u(t) - u(t-T)\\
					G_0(s) &= \frac{1}{s}-\frac{1}{s}e^{-sT} = \frac{1-e^{-sT}}{s}
				\end{align*}
		\end{column}
		\begin{column}{0.4\textwidth}
			\begin{figure}
				\centering
				\includegraphics[width=3cm]{images/zeroOrderHold.eps}
			\end{figure}
		\end{column}
	\end{columns}
\end{frame}

\begin{frame}[<+->]\frametitle{Transformada Zeta}
	\begin{itemize}
		\item Se utiliza en sistemas de datos muestreados para transformar una señal/sistema a una representación en el dominio de la frecuencia compleja.
		\item Es el equivalente de tiempo discreto de la transformada de Laplace.
		\item Se define como:
		\begin{equation*}
			\mathcal{Z}\left[ f(t) \right] = F(z) = \sum_{k=0}^{\infty}f(kT)z^{-k}
		\end{equation*}
		\item Ejemplo: Encontrar la transformada $\mathcal{Z}$ de un escalón unitario.
		\begin{align*}
			x(t) =
			\left\{
			\begin{array}{lr}
				0, & t < 0\\
				1, & t \geq 0
			\end{array}
			\right.
		\end{align*}
		\pause
		Para $t < 0$: $X(z) = \sum_{k=0}^{\infty} 0 z^{-k} = 0$\\
		\vspace*{3mm}
		\pause
		Para $t \geq 0$: $X(z) = \sum_{k=0}^{\infty} 1 z^{-k} = 1 + z^{-1} + z^{-2} + z^{-3} + \dots = \frac{1}{1-z^{-1}} = \frac{z}{z-1}$
	\end{itemize}
\end{frame}

\begin{frame}[c]\frametitle{Propiedades de la Transformada Zeta}
\begin{table}
	\begin{tabular}{cll}
		Prop & $x(t)$ & $X(z)$\\
		\hline
		1 & $kx(t)$ & $kX(z)$\\
		2 & $x_1(t) + x_2(t)$ & $X_1(z) + X_2(z)$\\
		3 & $x(t-nT)$ & $z^{-n}X(z)$, si $x(t) = 0$ para $t<0$ y $X(z)$ existe\\
		4 & $x(0)$, valor inicial & $\lim_{z\rightarrow \infty} X(z)$, si el límite existe\\
		5 & $x(\infty)$, valor final & $\lim_{z\rightarrow 1}(z-1) X(z)$, si el límite existe y el sistema es estable
	\end{tabular}
\end{table}
\end{frame}

\begin{frame}[c]\frametitle{Transformada Zeta de la Derivada de una Función}
	Encuentre la transformada Zeta de $\frac{dx(t)}{dt}$.
	\pause
	\begin{align*}
		&\frac{dx(t)}{dt} \approx \frac{x(kT) - x((k-1)T)}{T}\ \text{(approximación de primer orden)}\\
		&\mathcal{Z}\left\{\frac{x(kT) - x((k-1)T)}{T}\right\} = \frac{X(z)-z^{-1}X(z)}{T} = \frac{1-z^{-1}}{T} X(z)\\
		&= \frac{z-1}{zT} X(z)
	\end{align*}
	\vspace*{-3mm}
	\begin{columns}
		\pause
		\begin{column}{0.5\textwidth}
			Note que:
			\begin{align*}
				\mathcal{Z}\left\{\frac{dx(t)}{dt}\right\} &\approx \frac{1-z^{-1}}{T} X(z) \\
				\mathcal{L}\left\{\frac{dx(t)}{dt}\right\} &= s X(s)
			\end{align*}
		\end{column}
		\pause
		\begin{column}{0.5\textwidth}
		Entonces usando la aproximación de primer orden (backward):
		\begin{equation*}
			s \approx \frac{z-1}{zT}
		\end{equation*}
		\end{column}
	\end{columns}
\end{frame}

\begin{frame}[c]\frametitle{Transformada Zeta de la Integral de una Función}
	Usando la aproximación integral por trapecios:
	\begin{align*}
				\mathcal{Z}\left\{\int x(t) dt\right\} &\approx \frac{T}{2} \frac{1+z^{-1}}{1-z^{-1}} X(z) = \frac{T}{2} \frac{z+1}{z-1} X(z) \\
				\mathcal{L}\left\{\int x(t) dt\right\} &= \frac{1}{s} X(s)
	\end{align*}
	\pause
	Entonces, usando la aproximación trapezoidal (Tustin):
	\begin{equation*}
		s \approx \frac{2}{T}\frac{z-1}{z+1}
	\end{equation*}
\end{frame}

\begin{frame}[c]\frametitle{Discretización de Sistemas - Ejemplo}
	\vspace*{3mm}	
	Discretizar $G(s) = \frac{1}{s^2 + s + 1}$ para $T=1$ usando el método backward.\\
	\pause
	\vspace*{4mm}
	\begin{columns}
		\begin{column}{0.3\textwidth}
			Aproximación backward:
			\begin{equation*}
				s = \frac{z-1}{zT}
			\end{equation*}
			\pause
			\begin{align*}
				G(z) &= \frac{1}{\left(\frac{z-1}{z}\right)^2 + \frac{z-1}{z} + 1}\\
				&= \frac{z^3}{3z^3-3z^2+z}
			\end{align*}
		\end{column}
		\pause
		\begin{column}{0.7\textwidth}
			\lstinputlisting{code/backwardDiscretization.m}
		\end{column}
	\end{columns}
\end{frame}

\begin{frame}[c]\frametitle{Discretización de Sistemas - Ejemplo}
	\vspace*{3mm}	
	Discretizar $G(s) = \frac{1}{s^2 + s + 1}$ para $T=1$ usando el método backward.\\
	\begin{columns}
		\begin{column}{0.3\textwidth}
			Respuesta paso del sistema de tiempo continuo $G(s)$ y del sistema discretizado $G(z)$.
		\end{column}
		\begin{column}{0.7\textwidth}
			\begin{figure}
				\centering
				\includegraphics[width=8.5cm]{images/backwardStepResp.eps}
			\end{figure}
		\end{column}
	\end{columns}
\end{frame}

\begin{frame}[c]\frametitle{Discretización de Sistemas - Ejemplo}
	\vspace*{3mm}	
	Discretizar $G(s) = \frac{1}{s^2 + s + 1}$ para $T=1$ usando el método Tustin.\\
	\pause
	\vspace*{4mm}	
	\begin{columns}
		\begin{column}{0.3\textwidth}
			Approximación Tustin:
			\begin{equation*}
				s = \frac{2}{T}\frac{z-1}{z+1}
			\end{equation*}
			\pause
			\begin{align*}
				G(z) &= \frac{1}{\left(2\frac{z-1}{z+1}\right)^2 + 2\frac{z-1}{z+1} + 1}\\
				&= \frac{z^3+3z^2+3z+1}{7z^3+z^2-3z+3}
			\end{align*}
		\end{column}
		\pause
		\begin{column}{0.7\textwidth}
			\lstinputlisting{code/tustinDiscretization.m}
		\end{column}
	\end{columns}
\end{frame}

\begin{frame}[c]\frametitle{Discretización de Sistemas - Ejemplo}
	\vspace*{3mm}	
	Discretizar $G(s) = \frac{1}{s^2 + s + 1}$ para $T=1$ usando el método Tustin.\\
	\begin{columns}
		\begin{column}{0.3\textwidth}
			Respuesta paso del sistema de tiempo continuo $G(s)$ y del sistema discretizado $G(z)$.
		\end{column}
		\begin{column}{0.7\textwidth}
			\begin{figure}
				\centering
				\includegraphics[width=8.5cm]{images/tustinStepResp.eps}
			\end{figure}
		\end{column}
	\end{columns}
\end{frame}

\begin{frame}[c]\frametitle{Discretización de Sistemas - Ejemplo}
	\vspace*{3mm}	
	Discretizar $G(s) = \frac{1}{s^2 + s + 1}$ para $T = \{2,1,0.5,0.1\}$ usando el método backward.\\
	\begin{columns}
		\begin{column}{0.3\textwidth}
			Respuesta paso del sistema continuo $G(s)$ y de sistemas discretizados $G(z)$ con differentes periodos de muestreo $T$.
		\end{column}
		\begin{column}{0.7\textwidth}
			\begin{figure}
				\centering
				\includegraphics[width=8.5cm]{images/backwardStepRespDiffT.eps}
			\end{figure}
		\end{column}
	\end{columns}
\end{frame}

\begin{frame}[c]\frametitle{Discretización de Sistemas - Ejemplo}
	\vspace*{3mm}	
	Discretizar $G(s) = \frac{1}{s^2 + s + 1}$ para $T = \{2,1,0.5,0.1\}$ usando el método Tustin.\\
	\begin{columns}
		\begin{column}{0.3\textwidth}			
			Respuesta paso del sistema continuo $G(s)$ y de sistemas discretizados $G(z)$ con differentes periodos de muestreo $T$.
		\end{column}
		\begin{column}{0.7\textwidth}
			\begin{figure}
				\centering
				\includegraphics[width=8.5cm]{images/tustinStepRespDiffT.eps}
			\end{figure}
		\end{column}
	\end{columns}
\end{frame}

\begin{frame}[<+->]\frametitle{Procedimiento para el Diseño de Controladores Digitales}
	\begin{enumerate}
		\item Diseñar el controlador usando las técnicas para tiempo continuo.
		\item Seleccionar un método de discretización.
		\item Seleccionar el tiempo de muestreo $T$.
		\item Discretizar el controlador.
		\item Verificar que se satisfacen los requerimientos del sistema en lazo cerrado con el controlador discretizado.
		\item Expresar la función de transferencia de tiempo discreto como una ecuación de diferencias.
	\end{enumerate}
\end{frame}

\begin{frame}[c]\frametitle{Procedimiento para Diseño de Controladores Digitales - Ejemplo}
	Considere el sistema 
	\begin{equation*}
		G(s) = \frac{0.8}{(30s+1)(13s+1)(3s+1)}
	\end{equation*}
	Diseñe un algoritmo de control digital tal que el sistema:
	\begin{enumerate}
		\item Error de estado estacionario ante escalón unitario $e_{ss} = 0$.
		\item Porcentaje de sobrepico $PO \leq 15\%$.
		\item Tiempo de establecimiento $T_s \leq 100$.
	\end{enumerate}
\end{frame}

\begin{frame}[c]\frametitle{Procedimiento para Diseño de Controladores Digitales - Ejemplo}
	\begin{itemize}
		\item A partir de los requerimientos del problema se calcula que para el sistema de lazo cerrado $\zeta = 0.5165$, $\omega_n = 0.0774$.
		\pause
		El polinomio deseado correspondiente es $q(s) = s^2 +0.08s + 0.006$.
		\pause
		Los polos deseados son $s_{1,2} = -0.04 \pm 0.0663j$.
		\pause
		\item Se decide utilizar un controlador PID:
		\begin{equation*}
			G_c(s) = K_p \left(1 + \frac{1}{T_i s} + T_d s\right)
		\end{equation*}
		\pause
		Usando el método de LGR por aproximación de polos dominantes se encuentra que $K_p = 14.029$, $T_i = 13.5593$, $T_d = 12.5$.
	\end{itemize}
\end{frame}

\begin{frame}[c]\frametitle{Procedimiento para Diseño de Controladores Digitales - Ejemplo}
	\vspace*{1.7mm}
	\begin{columns}
		\begin{column}{0.3\textwidth}
			La respuesta paso del sistema en lazo cerrado con el controlador PID de tiempo continuo cumple con los requerimientos especificados.
		\end{column}
		\begin{column}{0.7\textwidth}
		\begin{figure}
			\centering
			\includegraphics[width=9cm]{images/stepRespPIDCont.eps}
		\end{figure}
		\end{column}
	\end{columns}
\end{frame}

\begin{frame}[c]\frametitle{Procedimiento para Diseño de Controladores Digitales - Ejemplo}
	\begin{columns}
		\begin{column}{0.3\textwidth}
			Modelo de Simulink empleado para comparar el desempeño de los controladores PID discretizados respecto al controlador PID de tiempo continuo.
		\end{column}
		\begin{column}{0.7\textwidth}
		\begin{figure}
			\centering
			\includegraphics[width=9cm]{images/simulinkPIDcomparison.eps}
		\end{figure}
		\end{column}
	\end{columns}
\end{frame}

\begin{frame}[c]\frametitle{Procedimiento para Diseño de Controladores Digitales - Ejemplo}
	\begin{columns}
		\begin{column}{0.3\textwidth}
			Comparación de respuestas paso del sistema en lazo cerrado con controlador PID de tiempo continuo y con controlador PID discretizado usando método backward con diferentes tiempos de muestreo.
		\end{column}
		\begin{column}{0.7\textwidth}
		\begin{figure}
			\centering
			\includegraphics[width=9cm]{images/stepRespPIDComparBackward.eps}
		\end{figure}
		\end{column}
	\end{columns}
\end{frame}

\begin{frame}[c]\frametitle{Procedimiento para Diseño de Controladores Digitales - Ejemplo}
	\begin{columns}
		\begin{column}{0.3\textwidth}
			Comparación de respuestas paso del sistema en lazo cerrado con controlador PID de tiempo continuo y con controlador PID discretizado usando método Tustin con diferentes tiempos de muestreo.
		\end{column}
		\begin{column}{0.7\textwidth}
		\begin{figure}
			\centering
			\includegraphics[width=9cm]{images/stepRespPIDComparTustin.eps}
		\end{figure}
		\end{column}
	\end{columns}
\end{frame}

\begin{frame}[c]\frametitle{Procedimiento para Diseño de Controladores Digitales - Ejemplo}
	Se decide utilizar el controlador discretizado usando el método de Tustin con un periodo de muestreo $T_s = 0.1$ s.
	\pause
	En este caso la función de transferencia del PID en tiempo discreto es:
	\begin{equation*}
		G_c(z) = \frac{U(z)}{E(z)} = \frac{9.549 \times 10^{5} z^2 - 1.902 \times  10^{6} z + 9.473 \times 10^{5}}{271.2 z^2 - 271.2}
	\end{equation*}
	\pause
	Asignando nombres a los coeficientes de la función de transferencia:
	\begin{equation*}
		G_c(z) = \frac{U(z)}{E(z)} = \frac{b_2 z^2 + b_1 z + b_0}{a_2 z^2 + a_0}
	\end{equation*}
	\pause
	Multiplicando el numerador y denominador por $z^{-2}$ se obtiene:
	\begin{equation*}
		\frac{U(z)}{E(z)} = \frac{b_2 + b_1 z^{-1} + b_0 z^{-2}}{a_2 + a_0 z^{-2}}
	\end{equation*}
\end{frame}

\begin{frame}[c]\frametitle{Procedimiento para Diseño de Controladores Digitales - Ejemplo}
	Reorganizando:
	\begin{align*}
		U(z)\left(a_2 + a_0 z^{-2}\right) = E(z)\left( b_2 + b_1 z^{-1} + b_0 z^{-2} \right)\\
		a_2 U(z) + a_0 z^{-2}U(z) = b_2 E(z) + b_1 z^{-1}E(z) + b_0 z^{-2}E(z)
	\end{align*}
	\pause
	Aplicando la transformada $\mathcal{Z}$ inversa:
	\begin{align*}
		a_2 u(k) + a_0 u(k-2) = b_2&e(k) + b_1 e(k-1) + b_0 e(k-2)
	\end{align*}
	\pause
	Despejando $u(k)$ se obtiene:
	\begin{equation*}
		u(k) = -\frac{a_0}{a_2}u(k-2) + \frac{b_2}{a_2} e(k) + \frac{b_1}{a_2} e(k-1) + \frac{b_0}{a_2} e(k-2)
	\end{equation*}
	\pause
	Esta ecuación de diferencias puede implementarse como un algoritmo recursivo que calcula el valor de $e$ en cada instante de tiempo discreto $k$ a partir de valores presentes y pasados de $e$ y $u$.
\end{frame}

\begin{frame}[c]\frametitle{Procedimiento para Diseño de Controladores Digitales - Ejemplo}
	\lstinputlisting[caption=Implementación del controlador PID discretizado como una ecuación de diferencias]{code/diffeq.m}
\end{frame}

\begin{frame}[c]\frametitle{Procedimiento para Diseño de Controladores Digitales - Ejemplo}
\begin{columns}
	\begin{column}{0.3\textwidth}
		Modelo de Simulink empleado para comparar el desempeño del control PID continuo, PID discreto como función de transferencia y PID discreto como ecuación de diferencias.
	\end{column}
	\begin{column}{0.7\textwidth}
		\begin{figure}
			\centering
			\includegraphics[width=9cm]{images/simulinkPIDTustin.eps}
		\end{figure}
	\end{column}
\end{columns}
\end{frame}

\begin{frame}[c]\frametitle{Procedimiento para Diseño de Controladores Digitales - Ejemplo}
\begin{columns}
	\begin{column}{0.3\textwidth}
		Señal de control $u(t)$ generada por el controlador PID continuo, PID discreto como función de transferencia y PID discreto como ecuación de diferencias.
	\end{column}
	\begin{column}{0.7\textwidth}
		\begin{figure}
			\centering
			\includegraphics[width=9cm]{images/stepRespPIDComparUTustin.eps}
		\end{figure}
	\end{column}
\end{columns}
\end{frame}
 
\begin{frame}[c]\frametitle{Procedimiento para Diseño de Controladores Digitales - Ejemplo}
\begin{columns}
	\begin{column}{0.3\textwidth}
		Señal de salida $y(t)$ de la planta obtenida usando el controlador PID continuo, PID discreto como función de transferencia y PID discreto como ecuación de diferencias.
	\end{column}
	\begin{column}{0.7\textwidth}
		\begin{figure}
			\centering
			\includegraphics[width=9cm]{images/stepRespPIDComparYTustin.eps}
		\end{figure}
	\end{column}
\end{columns}
\end{frame}

\begin{frame}[<+->]\frametitle{Conclusión}
	\begin{itemize}
		\item Control digital $\rightarrow$ alternativa interesante frente al control analógico.
		\item Ventajas:
		\begin{itemize}
			\item Mayor flexibilidad en la implementación respecto al control analógico.
			\item Menor susceptibilidad respecto al paso del tiempo y a las condiciones ambientales.
			\item Bajo costo - alto desempeño.
			\item Gran variedad de microcontroladores - microprocesadores con múltiples funciones y diseñados para diferentes aplicaciones.
		\end{itemize}
		\item Desventajas:
		\begin{itemize}
			\item Existencia de errores de aproximación en el paso de discretización $\rightarrow$ puede subsanarse diseñando el controlador digital directamente en el dominio discreto.
			\item Requiere usar retenedores (ZOH) que agregan retardos al lazo de control y pueden disminuir la estabilidad del sistema.
		\end{itemize}
	\end{itemize}
\end{frame}
 
\end{document}