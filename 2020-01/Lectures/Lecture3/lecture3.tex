\documentclass[aspectratio=169,handout]{beamer}
% \usepackage[utf8]{inputenc}
\usetheme{metropolis}
\usecolortheme{orchid}
\usepackage{amsmath}
\usepackage{amssymb}
\usepackage{amsthm}
\usepackage{multirow}
\usepackage[ruled]{algorithm2e}
\usepackage{mathtools}
\usepackage{caption}
\usepackage{epstopdf}
\usepackage{hyperref}
\setbeamerfont{footnote}{size=\tiny}

\usepackage{tikz}
\usetikzlibrary{mindmap,shadows,tikzmark,positioning,arrows.meta}

% Information boxes
\newcommand*{\info}[4][16.3]{%
  \node [ annotation, #3, scale=0.65, text width = #1em,
          inner sep = 2mm ] at (#2) {%
  \list{$\bullet$}{\topsep=0pt\itemsep=0pt\parsep=0pt
    \parskip=0pt\labelwidth=8pt\leftmargin=8pt
    \itemindent=0pt\labelsep=2pt}%
    #4
  \endlist
  };
}

\tikzset{%
  >={Latex[width=2mm,length=2mm]},
  % Specifications for style of nodes:
            base/.style = {rectangle, rounded corners, draw=black,
                           minimum width=4cm, minimum height=1cm,
                           text centered, font=\sffamily},
  activityStarts/.style = {base, fill=blue!30},
       startstop/.style = {base, fill=red!30},
    activityRuns/.style = {base, fill=green!30},
         process/.style = {base, minimum width=2.5cm, fill=orange!15,
                           font=\ttfamily},
}
\renewcommand\textbullet{\ensuremath{\bullet}}
\newcommand\scalemath[2]{\scalebox{#1}{\mbox{\ensuremath{\displaystyle #2}}}}
\newcommand{\norm}[1]{\left\lVert#1\right\rVert}

%%% Bibliography
\usepackage[citestyle=numeric,style=numeric,backend=biber,doi=false,isbn=false,url=false]{biblatex}
\addbibresource{references.bib}

%%% Suppress biblatex annoying warning
\usepackage{silence}
\WarningFilter{biblatex}{Patching footnotes failed}

%%% new theorems %%%%%%%%%%%%%%%%%%%%%%%%%%%%%%%%%%%%%%%%%%%%%%%%%%%%%%%%%%%%%%
\theoremstyle{definition}
\newtheorem{mydef}{Definition}

\theoremstyle{plain}
\newtheorem{mylemma}{Lemma}[section]
\newtheorem{mytheorem}{Theorem}[section]
\newtheorem{myproposition}{Proposition}[section]
\newtheorem{myproblem}{Problem}[section]
\newtheorem{mydefinition}{Definition}[section]
\newtheorem{myassumption}{Assumption}[section]

\theoremstyle{remark}
\newtheorem{myremark}{Remark}[section]

\newcounter{saveenumi}
\newcommand{\seti}{\setcounter{saveenumi}{\value{enumi}}}
\newcommand{\conti}{\setcounter{enumi}{\value{saveenumi}}}

\resetcounteronoverlays{saveenumi}

\title{Control de Sistemas}
\subtitle{\small Clase 3: Diagramas de Bloques de Sistemas en Lazo Cerrado}
\author{Gerardo Becerra, Ph.D.}
\institute{Pontificia Universidad Javeriana\\ Departamento de Electrónica}
\date{Febrero 11, 2020}

\begin{document}

\frame{\titlepage}	

% \frame{\tableofcontents}

\begin{frame}[<+->]\frametitle{Introducción}
\begin{itemize}
	\item Sistema: interconexión de subsistemas.
	\item Subsistema: representado por una función de transferencia.
	\item Objetivo: obtener la función de transferencia de un sistema formado por varios subsistemas.
	\item Representación gráfica de subsistemas:
	\begin{itemize}
		\item Diagramas de bloques.
		\item Diagramas de flujo.
	\end{itemize}
	\item Métodos para simplificar los diagramas:
	\begin{itemize}
		\item Diagramas de bloques: álgebra de bloques.
		\item Diagramas de flujo: regla de Mason.
	\end{itemize}
\end{itemize}
\end{frame}

\section{Diagramas de Bloques}
\begin{frame}[<+->]\frametitle{Diagramas de Bloques}
\textbf{Componentes:}\\
\vspace*{5mm}
\centering
\includegraphics[width=10cm]{images/blockDiagramElements.eps}
\begin{itemize}
	\item Aplican para sistemas lineales e invariantes en el tiempo (LTI).
	\item Pueden organizarse en múltiples configuraciones / topologías.
\end{itemize}
\end{frame}

\begin{frame}[<+->]\frametitle{Configuración en Cascada}
\begin{itemize}
	\item La salida de un subsistema se alimenta al siguiente subsistema.
	\begin{center}
		\includegraphics[width=10cm]{images/cascadeConfig1.eps}
	\end{center}
	\item La función de transferencia equivalente es el producto de las funciones de transferencia en cascada.
	\begin{center}
		\includegraphics[width=5cm]{images/cascadeConfig2.eps}
	\end{center}
	\item Éste resultado aplica bajo la suposición de que no existe \textbf{efecto de carga}: la salida de un subsistema se mantiene igual aunque el siguiente bloque se encuentre conectado o no.
\end{itemize}
\end{frame}

\begin{frame}[<+->]\frametitle{Efecto de Carga: Circuitos RC en Cascada}
\begin{columns}
	\begin{column}{0.5\textwidth}
		\centering
		\includegraphics[width=3.5cm]{images/loadEffect1.eps}
	\end{column}
	\pause
	\begin{column}{0.5\textwidth}
		\centering
		\includegraphics[width=6cm]{images/loadEffect2.eps}
	\end{column}	
\end{columns}
	\small	
	El efecto se previene usando un amplificador de ganancia $K$ con alta impedancia de entrada y baja impedancia de salida.
\end{frame}

\begin{frame}[<+->]\frametitle{Configuración Paralela}
	\begin{itemize}
		\item Varios subsistemas se alimentan con una misma entrada.
		\item Las salidas de los subsistemas se suman.
		\begin{center}
			\includegraphics[width=10cm]{images/parallelConfig1.eps}
		\end{center}
		\item La función de transferencia equivalente es la suma de las funciones de transferencia en paralelo.
		\begin{center}
			\includegraphics[width=6cm]{images/parallelConfig2.eps}
		\end{center}
	\end{itemize}
\end{frame}

\begin{frame}[<+->]\frametitle{Configuración en Lazo Retroalimentado}
	\begin{columns}
		\begin{column}{0.5\textwidth}
		\begin{itemize}
			\item Configuración fundamental en los sistemas de control.
			\item La salida se retroalimenta para compararla con la referencia y generar una señal de error.
		\end{itemize}
		\begin{center}
			\includegraphics[width=6cm]{images/feedbackConfig1.eps}
		\end{center}
		\end{column}
		\begin{column}{0.5\textwidth}
		\begin{itemize}
			\item La función de transferencia equivalente es:
			\begin{center}
				\includegraphics[width=4cm]{images/feedbackConfig2.eps}
			\end{center}
		\item El signo depende del tipo de retroalimentación (positiva o negativa).
		\end{itemize}
		\end{column}
	\end{columns}
\end{frame}

\begin{frame}[<+->]\frametitle{Movimiento de Bloques para Crear Formas Familiares}
	\begin{itemize}
		\item Formas familiares (cascada, paralelo, retroalimentación) no siempre son aparentes en el diagrama de bloques.
		\item Movimiento de bloques a través de puntos de unión o sumadores: permite obtener formas familiares.
	\end{itemize}
	\begin{columns}
		\begin{column}{0.5\textwidth}
			\centering
			\includegraphics[width=7cm]{images/blockMovement1.eps}
			\small Movimiento hacia atrás en un sumador.
		\end{column}	
		\pause
		\begin{column}{0.5\textwidth}
			\centering
			\includegraphics[width=7cm]{images/blockMovement2.eps}
			\small Movimiento hacia adelante en un sumador.
		\end{column}	
	\end{columns}
\end{frame}

\begin{frame}[<+->]\frametitle{Movimiento de Bloques para Crear Formas Familiares}
	\begin{columns}
		\begin{column}{0.5\textwidth}
			\centering
			\includegraphics[width=7cm]{images/blockMovement3.eps}
			\small Movimiento hacia atrás en un punto de unión.
		\end{column}	
		\pause
		\begin{column}{0.5\textwidth}
			\centering
			\includegraphics[width=7cm]{images/blockMovement4.eps}
			\small Movimiento hacia adelante en un punto de unión.
		\end{column}	
	\end{columns}
\end{frame}

\begin{frame}[<+->]\frametitle{Ejemplo 1: Reducción de Bloques usando Formas Familiares}
\textbf{Reducir el diagrama de bloques a una sola función de transferencia.}\\
\centering
\vspace*{5mm}
\includegraphics[width=9cm]{images/example1_familiarForms.eps}	
\end{frame}

\begin{frame}[<+->]\frametitle{Ejemplo 1: Reducción de Bloques usando Formas Familiares}
1. Colapsar todos los sumadores en uno sólo:\\
\centering
\vspace*{5mm}
\includegraphics[width=8cm]{images/example1_familiarForms1.eps}	
\end{frame}

\begin{frame}[<+->]\frametitle{Ejemplo 1: Reducción de Bloques usando Formas Familiares}
2. Encontrar el equivalente paralelo:\\
\centering
\vspace*{5mm}
\includegraphics[width=8cm]{images/example1_familiarForms2.eps}	
\end{frame}

\begin{frame}[<+->]\frametitle{Ejemplo 1: Reducción de Bloques usando Formas Familiares}
3. Calcular la función de transferencia total usando la fórmula de lazo retroalimentado:\\
\centering
\vspace*{5mm}
\includegraphics[width=8cm]{images/example1_familiarForms3.eps}	
\end{frame}

\begin{frame}[<+->]\frametitle{Ejemplo 2: Reducción de Bloques usando Movimiento de Bloques}
\textbf{Reducir el diagrama de bloques a una sola función de transferencia.}\\
\centering
\vspace*{5mm}
\includegraphics[width=11cm]{images/example2_blockMovement.eps}	
\end{frame}

\begin{frame}[<+->]\frametitle{Ejemplo 2: Reducción de Bloques usando Movimiento de Bloques}
\centering
\vspace*{5mm}
\includegraphics[width=11cm]{images/example2_blockMovement1.eps}	
\end{frame}

\begin{frame}[<+->]\frametitle{Ejemplo 2: Reducción de Bloques usando Movimiento de Bloques}
\centering
\vspace*{5mm}
\includegraphics[width=11cm]{images/example2_blockMovement2.eps}	
\end{frame}

\begin{frame}[<+->]\frametitle{Ejemplo 2: Reducción de Bloques usando Movimiento de Bloques}
\centering
\vspace*{5mm}
\includegraphics[width=11cm]{images/example2_blockMovement3.eps}	
\end{frame}

\begin{frame}[<+->]\frametitle{Ejemplo 2: Reducción de Bloques usando Movimiento de Bloques}
\centering
\vspace*{5mm}
\includegraphics[width=11cm]{images/example2_blockMovement4.eps}	
\end{frame}

\begin{frame}[<+->]\frametitle{Ejemplo 2: Reducción de Bloques usando Movimiento de Bloques}
\centering
\vspace*{5mm}
\includegraphics[width=11cm]{images/example2_blockMovement5.eps}	
\end{frame}

\section{Diagramas de Flujo}
\begin{frame}[<+->]\frametitle{Diagramas de Flujo}
\begin{columns}
\begin{column}{0.5\textwidth}
\begin{itemize}
	\item Alternativa a los diagramas de flujo.
	\item Elementos:
	\begin{itemize}
		\item Ramas: representan sistemas.
		\item Nodos: representan señales.
	\end{itemize}
	\item Las ramas tienen una flecha. Representan la dirección de flujo de la señal a través del sistema.
	\item Cada señal es igual a la suma de señales que entran al nodo.
\end{itemize}
\end{column}	
\begin{column}{0.5\textwidth}
\centering
\vspace*{5mm}
\includegraphics[width=7cm]{images/flowDiagrams.eps}	
\end{column}	
\end{columns}
\end{frame}

\begin{frame}[<+->]\frametitle{Ejemplo 3: Convertir Diagramas de Bloques en Diagramas de Flujo}
\textbf{Convierta los diagramas de bloques de las formas familiares a diagramsa de flujo.}\\
\end{frame}

\begin{frame}[<+->]\frametitle{Ejemplo 4: Convertir Diagramas de Bloques en Diagramas de Flujo}
\textbf{Convierta el siguiente diagrama de bloques en diagrama de flujo.}\\
\centering
\vspace*{5mm}
\includegraphics[width=11cm]{images/example2_blockMovement.eps}	
\end{frame}

\begin{frame}[<+->]\frametitle{Regla de Mason}
\begin{itemize}
	\item Aplicación de una fórmula obtenida por S.J. Mason (1953).
	\item Puede ser más fácil obtener la función de transferencia que usando reducción de diagramas de bloques.
	\item Definición de elementos del diagrama:
	\begin{itemize}
		\item Ganancia de lazo: Producto de ganancias de ramas que se recorren iniciando en un nodo y finalizando en el mismo nodo siguiendo la dirección de flujo, sin pasar por cualquier otro nodo más de una vez.
		\item Ganancia de trayectoria: Producto de ganancias de ramas que se recorren desde el nodo de entrada hasta el nodo de salida siguiendo la dirección de flujo.
		\item Lazos que no se tocan: Lazos que no tienen nodos en común.
	\end{itemize}
\end{itemize}
\end{frame}

\begin{frame}[<+->]\frametitle{Regla de Mason}
\begin{theorem}
\footnotesize
La función de transferencia $C(s)/R(s)$	de un sistema representado por un diagrama de flujo es:
\begin{equation*}
	G(s) = \frac{C(s)}{R(s)} = \frac{\sum_k T_k \Delta_k}{\Delta}		
\end{equation*}	
Donde:
\begin{itemize}
	\item $k$: Número de trayectorias directas.
	\item $T_k$: Ganancia de la $k$-ésima trayectoria directa.
	\item $\Delta$ = 1 - $\sum$ ganancias de lazo individuales + $\sum$ ganancias de 2 lazos que no se tocan - $\sum$ ganancias de 3 lazos que no se tocan + $\sum$ ganancias de 4 lazos que no se tocan - ...
	\item $\Delta_k$ = 1 - $\sum$ ganancias de lazo en $\Delta$ que tocan la $k$-ésima trayectoria. ($\Delta_k$ se forma eliminando de $\Delta$ las ganancias de lazo que tocan la $k$-ésima trayectoria).
\end{itemize}
\end{theorem}
\end{frame}

\begin{frame}[<+->]\frametitle{Taller}
\begin{columns}
\begin{column}{0.5\textwidth}
\centering
\vspace*{5mm} \includegraphics[width=8cm]{images/exercise1.eps}
\end{column}
\begin{column}{0.5\textwidth}
Para el diagrama de bloques de la figura:\\
\begin{itemize}
	\item Obtenga la función de transferencia usando reducción de bloques.
	\item Obtenga el diagrama de flujo equivalente.
	\item Obtenta la función de transferencia usando la regla de Mason.
	\item Verifique los resultados y concluya.
\end{itemize}
\end{column}
\end{columns}
\end{frame}

% \begin{frame}[<+->]\frametitle{Modelamiento de Sistemas - Procedimiento}
% \begin{enumerate}
%   \item Definir el sistema y sus componentes.
%   \item Formular las relaciones básicas entre variables y suposiciones usando los principios fundamentales.
%   \item Obtener las ecuaciones diferenciales que representan el modelo matemático.
%   \item Solucionar las ecuaciones para las variables deseadas.
%   \item Examinar las soluciones y las suposiciones.
%   \item En caso necesario, analizar o diseñar el modelo nuevamente.
% \end{enumerate}
% \end{frame}

% \begin{frame}[<+->]\frametitle{Ecuaciones Diferenciales de Sistemas Físicos}
%   \begin{itemize}
%     \item Describen el comportamiento dinámico de los sistemas.
%     \item Se obtienen utilizando los principios físicos de los procesos.
%     \item Se utilizan leyes de interconexión para definir la interacción entre elementos/subsistemas.
%     \item Ecuaciones diferenciales lineales $\rightarrow$ funciones de transferencia.
%     \item Ecuaciones diferenciales no lineales $\rightarrow$ linealización.
%   \end{itemize}
% \end{frame}

% \begin{frame}[<+->]\frametitle{Ecuaciones Diferenciales de Sistemas Físicos}
%   \textbf{Variables y Parámetros:}
%   \begin{itemize}
%     \item Variables pasantes: $F$ (fuerza), $T$ (torque), $i$ (corriente), $Q$ (flujo volumétrico), $q$ (flujo de calor).
%     \item Variables transversales: $v$ (velocidad traslacional), $\omega$ (velocidad angular), $V$ (voltaje), $P$ (presión), $\mathcal{T}$ (temperatura).
%     \item Almacenamiento inductivo: $L$ (inductancia), $1/k$ (rigidez traslacional o rotacional inversa), $I$ (inertancia).
%     \item Almacenamiento capacitivo: $C$ (capacitancia), $M$ (masa), $J$ (momento de inercia), $C_f$ (capacitancia de fluido), $C_t$ (capacitancia térmica).
%     \item Disipación de energía: $R$ (resistencia), $b$ (fricción viscosa), $R_f$ (resistencia de fluido), $R_f$ (resistencia térmica).
%   \end{itemize}
% \end{frame}

% \begin{frame}[<+->]\frametitle{Ecuaciones Diferenciales de Sistemas Físicos - Relaciones Fundamentales}
% \centering
% \includegraphics[width=14cm]{images/inductance.eps}
% \end{frame}

% \begin{frame}[<+->]\frametitle{Ecuaciones Diferenciales de Sistemas Físicos - Relaciones Fundamentales}
% \centering
% \includegraphics[width=14cm]{images/capacitance.eps}
% \end{frame}

% \begin{frame}[<+->]\frametitle{Ecuaciones Diferenciales de Sistemas Físicos - Relaciones Fundamentales}
% \centering
% \includegraphics[width=14cm]{images/disipation.eps}
% \end{frame}

% \begin{frame}[<+->]\frametitle{Modelos de Sistemas - Ejemplo 1}
% \centering
% \includegraphics[width=9cm]{images/circuit1.eps}
% \begin{itemize}
%   \item Obtener las ecuaciones diferenciales que describen el sistema.
%   \item Obtener la representación en variables de estado.
%   \item Obtener la función de transferencia.
% \end{itemize}
% \end{frame}

% \begin{frame}[<+->]\frametitle{Modelos de Sistemas - Ejemplo 1}
% \textbf{Ecuaciones diferenciales:}
% \begin{subequations}\label{eq:diff_eqs}
% \begin{align}
%   \frac{di_{L_1}(t)}{dt} &= -\frac{R_1}{L_1}i_{L_1}(t) - \frac{1}{L_1} v_c(t) + \frac{1}{L_1} v_i(t) \\
%   \frac{di_{L_2}(t)}{dt} &= -\frac{R_2}{L_2}i_{L_2}(t) + \frac{1}{L_2} v_c(t) \\
%   \frac{dv_c(t)}{dt} &= \frac{1}{C}i_{L_1}(t) - \frac{1}{C}i_{L_2}(t)
% \end{align}
% \textbf{Ecuación de salida:}
% \begin{equation}
%   V_o(t) = R_2 i_{L_2}(t)
% \end{equation}
% \end{subequations}
% \end{frame}

% \begin{frame}[<+->]\frametitle{Modelos de Sistemas - Ejemplo 1}
% \textbf{Diagrama de bloques y respuesta paso:}\\
% \vspace*{3mm}
% \centering
% \includegraphics[width=6.5cm]{code/example1_blockdiagram.eps}
% \pause
% \hspace*{5mm}
% \includegraphics[width=6.5cm]{code/example1_stepresponse.eps}
% \end{frame}

% \begin{frame}[<+->]\frametitle{Modelos de Sistemas - Ejemplo 1}
% Organizando las ecs. \eqref{eq:diff_eqs} en forma matricial:
% \begin{subequations}
% \begin{align}
%   \begin{bmatrix}
%    \frac{di_{L_1}(t)}{dt} \\ \frac{di_{L_2}(t)}{dt} \\ \frac{dv_c(t)}{dt} 
%   \end{bmatrix} &= 
%   \begin{bmatrix}
%   -\frac{R_1}{L_1} & 0 & -\frac{1}{L_1} \\
%   0 & -\frac{R_2}{L_2} &  \frac{1}{L_2} \\
%   \frac{1}{C} & -\frac{1}{C} & 0
%   \end{bmatrix}
%   \begin{bmatrix}
%    i_{L_1}(t) \\ i_{L_2}(t) \\ v_c(t) 
%   \end{bmatrix} + 
%   \begin{bmatrix}
%     \frac{1}{L_1} \\ 0 \\ 0
%   \end{bmatrix} v_i(t)\\
%   v_o(t) &=
%   \begin{bmatrix}
%     0 & R_2 & 0
%   \end{bmatrix}
%   \begin{bmatrix}
%    i_{L_1}(t) \\ i_{L_2}(t) \\ v_c(t) 
%   \end{bmatrix} + 
%   \begin{bmatrix}
%     0
%   \end{bmatrix} v_i(t)
% \end{align}
% \end{subequations}
% se obtiene la \textbf{representación en variables de estado:}
% \begin{subequations}
%   \begin{align}
%     \dot{x}(t) &= \mathbf{A} x(t) + \mathbf{B}u(t)\\
%     y(t) &= \mathbf{C}x(t) + \mathbf{D}u(t)
%   \end{align}
% \end{subequations}
% \end{frame}

% \begin{frame}[<+->]\frametitle{Modelos de Sistemas}
%   \begin{align*}
%     \dot{x}(t) &= \mathbf{A} x(t) + \mathbf{B}u(t)\\
%     y(t) &= \mathbf{C}x(t) + \mathbf{D}u(t)
%   \end{align*}
% \textbf{Estado de un Sistema:}
% \begin{itemize}
%   \item Mínima cantidad de información que, junto con la entrada, determinan la respuesta del sistema.
%   \item Resume la información pasada requerida para determinar el comportamiento futuro.
%   \item Se definen en sistemas con almacenamiento de energía: no aplican para sistemas instantáneos.
% \end{itemize}
% \end{frame}

% \begin{frame}[<+->]\frametitle{Modelos de Sistemas - Ejemplo 1}
% Aplicando la transformada de Laplace a las Ecs. \eqref{eq:diff_eqs}:
% \begin{subequations}\label{eq:algebraic_eqs}
% \begin{align}
%   s I_{L_1}(s) &= -\frac{R_1}{L_1}I_{L_1}(s) - \frac{1}{L_1} V_c(s) + \frac{1}{L_1} V_i(s) \\
%   s i_{L_2}(s) &= -\frac{R_2}{L_2}I_{L_2}(s) + \frac{1}{L_2} V_c(s) \\
%   s V_c(s) &= \frac{1}{C}I_{L_1}(s) - \frac{1}{C}I_{L_2}(s)\\
%   V_o(s) &= R_2 I_{L_2}(s)
% \end{align}
% \end{subequations}
% \vspace*{-5mm}
% \begin{itemize}
%   \item \textbf{Objetivo:} A partir de las Ecs. \eqref{eq:algebraic_eqs}, obtener $V_o(s)/V_i(s)$.
%   \item \textbf{Procedimiento:} Manipulación Algebráica.
%   \item \textbf{Alternativa:} Usando la representación en variables de estado.
% \end{itemize}
% \end{frame}

% \begin{frame}[<-+>]\frametitle{Modelos de Sistemas - Ejemplo 1}
%   Usando la fórmula:
%   \begin{equation}
%     H(s) = \mathbf{C}(s\mathbf{I}-\mathbf{A})^{-1}\mathbf{B} + \mathbf{D}
%   \end{equation}
%   se obtiene la función de transferencia:
%   \begin{equation}
%     H(s) = \frac{2\times10^{11}}{s^3 + 5.1\times10^4 s^2 + 5.6\times10^7 s + 2.02\times10^{11}}
%   \end{equation}
% \end{frame}

% \section{Tipos de Respuesta Transitoria}
% \begin{frame}[<-+>]\frametitle{Tipos de Sistemas}
% \begin{itemize}
%   \item Sistemas de \textbf{primer orden}:
%   \begin{equation*}
%     H(s) = \frac{K}{\tau s + 1}
%   \end{equation*}
%   \item Sistemas de \textbf{primer orden con tiempo muerto}:
%   \begin{equation*}
%     H(s) = \frac{K e^{-Ls}}{\tau s + 1}
%   \end{equation*}
%   \item Sistemas de \textbf{segundo orden}:
%   \begin{equation*}
%     H(s) = \frac{\omega_n^2}{s^2 + 2 \zeta \omega_n s + \omega_n^2}
%   \end{equation*}
% \end{itemize}
% \end{frame}

% \begin{frame}[<+->][c]\frametitle{Sistemas de Primer Orden - Respuesta Paso}
% \vspace*{3mm}
% \begin{columns}
%  \begin{column}{0.3\textwidth}
%   \begin{equation*}
%     H(s) = \frac{K}{\tau s + 1}
%   \end{equation*}
%   $K$: Ganancia\\
%   $\tau$: Constante de tiempo
%  \end{column} 
%  \begin{column}{0.7\textwidth}
%   \centering
%   \includegraphics[width=10cm]{images/firstOrderResponse.eps}
%  \end{column} 
% \end{columns}
% \end{frame}

% \begin{frame}\frametitle{Sistemas de Primer Orden mas Tiempo Muerto - Respuesta Paso}
% \vspace*{3mm}
% \begin{columns}
%  \begin{column}{0.3\textwidth}
%   \begin{equation*}
%     H(s) = \frac{K e^{-Ls}}{\tau s + 1}
%   \end{equation*}
%   $K$: Ganancia\\
%   $\tau$: Constante de tiempo\\
%   $L$: Tiempo muerto
%  \end{column} 
%  \begin{column}{0.7\textwidth}
%   \centering
%   \includegraphics[width=10cm]{images/firstOrder+deadTimeResponse.eps}
%  \end{column} 
% \end{columns}
% \end{frame}

% \begin{frame}\frametitle{Sistemas de Segundo Orden - Respuesta Paso}
% \vspace*{3mm}
% \begin{columns}
%  \begin{column}{0.3\textwidth}
%   \begin{equation*}
%     H(s) = \frac{\omega_n^2}{s^2 + 2 \zeta \omega_n s + \omega_n^2}
%   \end{equation*}
%   $\omega_n$: Frecuencia natural\\
%   $\zeta$: Factor de amortiguamiento\\
%   \vspace*{5mm}
%   $\zeta < 1$: subamortiguado\\
%   $\zeta = 1$: críticamente amortiguado\\
%   $\zeta > 1$: sobreamortiguado

%  \end{column} 
%  \begin{column}{0.7\textwidth}
%   \centering
%   \includegraphics[width=9cm]{images/secondOrderResponse.eps}
%  \end{column} 
% \end{columns}
% \end{frame}

% \begin{frame}[<+->]\frametitle{Respuesta Transitoria}
%   La respuesta transitoria del sistema puede describirse en función de dos factores:
%   \begin{itemize}
%     \item La \textbf{rapidez de la respuesta}, la cual está representada por el \textbf{tiempo de subida} y el \textbf{tiempo pico}.
%     \item La \textbf{proximidad de la respuesta al valor final}, representada por el \textbf{sobrepico} y el \textbf{tiempo de establecimiento}.
%   \end{itemize}
% \end{frame}

% \begin{frame}[<+->]\frametitle{Respuesta Transitoria}
%   \begin{itemize}
%     \item \textbf{Tiempo de subida} ($T_r$): Tiempo que tarda la respuesta en ir del 10\% al 90\% del valor final.
%     \item \textbf{Tiempo pico} ($T_p$): Tiempo en el cual la respuesta alcanza el valor máximo.
%     \item \textbf{Sobrepico} ($PO$): Relación en porcentaje entre el valor máximo y el valor final.
%     \item \textbf{Tiempo de establecimiento} ($T_s$): Tiempo que tarda la respuesta en mantenerse dentro de un 2\% del valor final.
%   \end{itemize}
% \end{frame}

% \begin{frame}\frametitle{Respuesta Transitoria - Sistemas de Primer Orden}
% \vspace*{3mm}
% \begin{columns}
%  \begin{column}{0.3\textwidth}
%  \textbf{Tiempo de subida:}
%  \begin{equation*}
%    T_r = 2.2 \tau
%  \end{equation*}
%  \textbf{Tiempo de establecimiento:}
%  \begin{equation*}
%   T_s = 4 \tau
%  \end{equation*}
%  \end{column} 
%  \begin{column}{0.7\textwidth}
%   \centering
%   \includegraphics[width=10cm]{images/firstOrderResponse.eps}
%  \end{column} 
% \end{columns}
% \end{frame}

% \begin{frame}[<-+>]\frametitle{Respuesta Transitoria - Sistemas de Primer Orden mas Tiempo Muerto}
% \vspace*{3mm}
% \begin{columns}
%  \begin{column}{0.3\textwidth}
%  \textbf{Tiempo de subida:}
%  \begin{equation*}
%    T_r = 2.2 \tau
%  \end{equation*}
%  \textbf{Tiempo de establecimiento:}
%  \begin{equation*}
%   T_s = L + 4 \tau
%  \end{equation*}
%  \end{column} 
%  \begin{column}{0.7\textwidth}
%   \centering
%   \includegraphics[width=10cm]{images/firstOrder+deadTimeResponse.eps}
%  \end{column} 
% \end{columns}
% \end{frame}

% \begin{frame}\frametitle{Respuesta Transitoria - Sistemas de Segundo Orden}
% \vspace*{3mm}
% \begin{columns}
%  \begin{column}{0.3\textwidth}
%  \textbf{Tiempo de establecimiento:}
%  \begin{equation*}
%   T_s = 4 \tau = \frac{4}{\zeta \omega_n}
%  \end{equation*}
%  \textbf{Tiempo de pico:}
%  \begin{equation*}
%    T_p = \frac{\pi}{\omega_n \sqrt{1-\zeta^2}}
%  \end{equation*}
%  \textbf{Sobrepico:}
%  \begin{equation*}
%    PO = 100 e^{-\zeta \pi/\sqrt{1-\zeta^2}}
%  \end{equation*}
%  \end{column} 
%  \begin{column}{0.7\textwidth}
%   \centering
%   \includegraphics[width=9cm]{images/secondOrderResponseParameters.eps}
%  \end{column} 
% \end{columns}
% \end{frame}

% \begin{frame}\frametitle{Respuesta Transitoria - Sistemas de Segundo Orden}
% \vspace*{3mm}
% \begin{columns}
%  \begin{column}{0.3\textwidth}
%  \textbf{Tiempo de subida:}
%  \begin{equation*}
%    T_{r1} = \frac{2.16\zeta + 0.60}{\omega_n}
%  \end{equation*}
%  Aproximación lineal válida para $0.3 \leq \zeta \leq 0.8$.
%  \end{column} 
%  \begin{column}{0.7\textwidth}
%   \centering
%   \includegraphics[width=9cm]{images/secondOrderRiseTime.eps}
%  \end{column} 
% \end{columns}
% \end{frame}

% \begin{frame}\frametitle{Respuesta Transitoria - Sistemas de Segundo Orden}
% \vspace*{3mm}
% \centering
% \includegraphics[width=8cm]{images/secondOrderResponseSameZeta.eps}\\
% Respuesta paso para $\zeta = 0.2$ con $\omega_n = 1$ y $\omega_n = 10$.
% \end{frame}

% \begin{frame}\frametitle{Respuesta Transitoria - Sistemas de Segundo Orden}
% \vspace*{3mm}
% \centering
% \includegraphics[width=8cm]{images/secondOrderResponseSameOmega.eps}\\
% Respuesta paso para $\omega_n = 5$ con $\zeta = 0.7$ y $\zeta = 1$.
% \end{frame}

% \begin{frame}[c]\frametitle{Taller}
% \begin{enumerate}
%   \item Para el sistema mostrado en la figura, considere como entrada la fuerza aplicada sobre la masa de la izquierda, y la salida como la distancia entre las dos masas. Obtenga las ecuaciones diferenciales que describen el sistema, la representación en el espacio de estados y la función de transferencia.
%   \seti
% \end{enumerate}
% \begin{center}
%   \includegraphics[width=8cm]{images/twoMassSystem.eps}
% \end{center}
% \end{frame}

% \begin{frame}[c]\frametitle{Taller}
% \begin{enumerate}
%   \conti
%   \item El control de inyecciones de insulina puede permitir mejorar la calidad de vida de pacientes diabéticos. La inyección automática de insulina usando una bomba y un sensor que mide los niveles de azucar en la sangre puede ser un tratamiento efectivo. La figura muestra el sistema de control correspondiente a éste proceso. Calcule un valor apropiado para $K$ tal que $PO = 7\%$. Calcule $T_s$ y $T_p$.
%   \seti
% \end{enumerate}
% \begin{center}
%   \includegraphics[width=10cm]{images/insulin.eps}
% \end{center}
% \end{frame}


\end{document}