\documentclass[aspectratio=169,handout]{beamer}
% \usepackage[utf8]{inputenc}
\usetheme{metropolis}
\usecolortheme{orchid}
\usepackage{amsmath}
\usepackage{amssymb}
\usepackage{amsthm}
\usepackage{multirow}
\usepackage[ruled]{algorithm2e}
\usepackage{mathtools}
\usepackage{caption}
\usepackage{epstopdf}
\usepackage{hyperref}
\setbeamerfont{footnote}{size=\tiny}

\usepackage{tikz}
\usetikzlibrary{mindmap,shadows,tikzmark,positioning,arrows.meta}

% Information boxes
\newcommand*{\info}[4][16.3]{%
  \node [ annotation, #3, scale=0.65, text width = #1em,
          inner sep = 2mm ] at (#2) {%
  \list{$\bullet$}{\topsep=0pt\itemsep=0pt\parsep=0pt
    \parskip=0pt\labelwidth=8pt\leftmargin=8pt
    \itemindent=0pt\labelsep=2pt}%
    #4
  \endlist
  };
}

\tikzset{%
  >={Latex[width=2mm,length=2mm]},
  % Specifications for style of nodes:
            base/.style = {rectangle, rounded corners, draw=black,
                           minimum width=4cm, minimum height=1cm,
                           text centered, font=\sffamily},
  activityStarts/.style = {base, fill=blue!30},
       startstop/.style = {base, fill=red!30},
    activityRuns/.style = {base, fill=green!30},
         process/.style = {base, minimum width=2.5cm, fill=orange!15,
                           font=\ttfamily},
}
\renewcommand\textbullet{\ensuremath{\bullet}}
\newcommand\scalemath[2]{\scalebox{#1}{\mbox{\ensuremath{\displaystyle #2}}}}
\newcommand{\norm}[1]{\left\lVert#1\right\rVert}

%%% Bibliography
\usepackage[citestyle=numeric,style=numeric,backend=biber,doi=false,isbn=false,url=false]{biblatex}
\addbibresource{references.bib}

%%% Suppress biblatex annoying warning
\usepackage{silence}
\WarningFilter{biblatex}{Patching footnotes failed}

%%% new theorems %%%%%%%%%%%%%%%%%%%%%%%%%%%%%%%%%%%%%%%%%%%%%%%%%%%%%%%%%%%%%%
\theoremstyle{definition}
\newtheorem{mydef}{Definition}

\theoremstyle{plain}
\newtheorem{mylemma}{Lemma}[section]
\newtheorem{mytheorem}{Theorem}[section]
\newtheorem{myproposition}{Proposition}[section]
\newtheorem{myproblem}{Problem}[section]
\newtheorem{mydefinition}{Definition}[section]
\newtheorem{myassumption}{Assumption}[section]

\theoremstyle{remark}
\newtheorem{myremark}{Remark}[section]

\newcounter{saveenumi}
\newcommand{\seti}{\setcounter{saveenumi}{\value{enumi}}}
\newcommand{\conti}{\setcounter{enumi}{\value{saveenumi}}}

\resetcounteronoverlays{saveenumi}

\title{Control de Sistemas}
\subtitle{\small Clase 3: Diagramas de Bloques de Sistemas en Lazo Cerrado}
\author{Gerardo Becerra, Ph.D.}
\institute{Pontificia Universidad Javeriana\\ Departamento de Electrónica}
\date{Febrero 11, 2020}

\begin{document}

\frame{\titlepage}	

% \frame{\tableofcontents}

\begin{frame}[<+->]\frametitle{Introducción}
\begin{itemize}
	\item Sistema: interconexión de subsistemas.
	\item Subsistema: representado por una función de transferencia.
	\item Objetivo: obtener la función de transferencia de un sistema formado por varios subsistemas.
	\item Representación gráfica de subsistemas:
	\begin{itemize}
		\item Diagramas de bloques.
		\item Diagramas de flujo.
	\end{itemize}
	\item Métodos para simplificar los diagramas:
	\begin{itemize}
		\item Diagramas de bloques: álgebra de bloques.
		\item Diagramas de flujo: regla de Mason.
	\end{itemize}
\end{itemize}
\end{frame}

\section{Diagramas de Bloques}
\begin{frame}[<+->]\frametitle{Diagramas de Bloques}
\textbf{Componentes:}\\
\vspace*{5mm}
\centering
\includegraphics[width=10cm]{images/blockDiagramElements.eps}
\begin{itemize}
	\item Aplican para sistemas lineales e invariantes en el tiempo (LTI).
	\item Pueden organizarse en múltiples configuraciones / topologías.
\end{itemize}
\end{frame}

\begin{frame}[<+->]\frametitle{Configuración en Cascada}
\begin{itemize}
	\item La salida de un subsistema se alimenta al siguiente subsistema.
	\begin{center}
		\includegraphics[width=10cm]{images/cascadeConfig1.eps}
	\end{center}
	\item La función de transferencia equivalente es el producto de las funciones de transferencia en cascada.
	\begin{center}
		\includegraphics[width=5cm]{images/cascadeConfig2.eps}
	\end{center}
	\item Éste resultado aplica bajo la suposición de que no existe \textbf{efecto de carga}: la salida de un subsistema se mantiene igual aunque el siguiente bloque se encuentre conectado o no.
\end{itemize}
\end{frame}

\begin{frame}[<+->]\frametitle{Efecto de Carga: Circuitos RC en Cascada}
\begin{columns}
	\begin{column}{0.5\textwidth}
		\centering
		\includegraphics[width=3.5cm]{images/loadEffect1.eps}
	\end{column}
	\pause
	\begin{column}{0.5\textwidth}
		\centering
		\includegraphics[width=6cm]{images/loadEffect2.eps}
	\end{column}	
\end{columns}
	\small	
	El efecto se previene usando un amplificador de ganancia $K$ con alta impedancia de entrada y baja impedancia de salida.
\end{frame}

\begin{frame}[<+->]\frametitle{Configuración Paralela}
	\begin{itemize}
		\item Varios subsistemas se alimentan con una misma entrada.
		\item Las salidas de los subsistemas se suman.
		\begin{center}
			\includegraphics[width=10cm]{images/parallelConfig1.eps}
		\end{center}
		\item La función de transferencia equivalente es la suma de las funciones de transferencia en paralelo.
		\begin{center}
			\includegraphics[width=6cm]{images/parallelConfig2.eps}
		\end{center}
	\end{itemize}
\end{frame}

\begin{frame}[<+->]\frametitle{Configuración en Lazo Retroalimentado}
	\begin{columns}
		\begin{column}{0.5\textwidth}
		\begin{itemize}
			\item Configuración fundamental en los sistemas de control.
			\item La salida se retroalimenta para compararla con la referencia y generar una señal de error.
		\end{itemize}
		\begin{center}
			\includegraphics[width=6cm]{images/feedbackConfig1.eps}
		\end{center}
		\end{column}
		\begin{column}{0.5\textwidth}
		\begin{itemize}
			\item La función de transferencia equivalente es:
			\begin{center}
				\includegraphics[width=4cm]{images/feedbackConfig2.eps}
			\end{center}
		\item El signo depende del tipo de retroalimentación (positiva o negativa).
		\end{itemize}
		\end{column}
	\end{columns}
\end{frame}

\begin{frame}[<+->]\frametitle{Movimiento de Bloques para Crear Formas Familiares}
	\begin{itemize}
		\item Formas familiares (cascada, paralelo, retroalimentación) no siempre son aparentes en el diagrama de bloques.
		\item Movimiento de bloques a través de puntos de unión o sumadores: permite obtener formas familiares.
	\end{itemize}
	\begin{columns}
		\begin{column}{0.5\textwidth}
			\centering
			\includegraphics[width=7cm]{images/blockMovement1.eps}
			\small Movimiento hacia atrás en un sumador.
		\end{column}	
		\pause
		\begin{column}{0.5\textwidth}
			\centering
			\includegraphics[width=7cm]{images/blockMovement2.eps}
			\small Movimiento hacia adelante en un sumador.
		\end{column}	
	\end{columns}
\end{frame}

\begin{frame}[<+->]\frametitle{Movimiento de Bloques para Crear Formas Familiares}
	\begin{columns}
		\begin{column}{0.5\textwidth}
			\centering
			\includegraphics[width=7cm]{images/blockMovement3.eps}
			\small Movimiento hacia atrás en un punto de unión.
		\end{column}	
		\pause
		\begin{column}{0.5\textwidth}
			\centering
			\includegraphics[width=7cm]{images/blockMovement4.eps}
			\small Movimiento hacia adelante en un punto de unión.
		\end{column}	
	\end{columns}
\end{frame}

\begin{frame}[<+->]\frametitle{Ejemplo 1: Reducción de Bloques usando Formas Familiares}
\textbf{Reducir el diagrama de bloques a una sola función de transferencia.}\\
\centering
\vspace*{5mm}
\includegraphics[width=9cm]{images/example1_familiarForms.eps}	
\end{frame}

\begin{frame}[<+->]\frametitle{Ejemplo 1: Reducción de Bloques usando Formas Familiares}
1. Colapsar todos los sumadores en uno sólo:\\
\centering
\vspace*{5mm}
\includegraphics[width=8cm]{images/example1_familiarForms1.eps}	
\end{frame}

\begin{frame}[<+->]\frametitle{Ejemplo 1: Reducción de Bloques usando Formas Familiares}
2. Encontrar el equivalente paralelo:\\
\centering
\vspace*{5mm}
\includegraphics[width=8cm]{images/example1_familiarForms2.eps}	
\end{frame}

\begin{frame}[<+->]\frametitle{Ejemplo 1: Reducción de Bloques usando Formas Familiares}
3. Calcular la función de transferencia total usando la fórmula de lazo retroalimentado:\\
\centering
\vspace*{5mm}
\includegraphics[width=8cm]{images/example1_familiarForms3.eps}	
\end{frame}

\begin{frame}[<+->]\frametitle{Ejemplo 2: Reducción de Bloques usando Movimiento de Bloques}
\textbf{Reducir el diagrama de bloques a una sola función de transferencia.}\\
\centering
\vspace*{5mm}
\includegraphics[width=11cm]{images/example2_blockMovement.eps}	
\end{frame}

\begin{frame}[<+->]\frametitle{Ejemplo 2: Reducción de Bloques usando Movimiento de Bloques}
\centering
\vspace*{5mm}
\includegraphics[width=11cm]{images/example2_blockMovement1.eps}	
\end{frame}

\begin{frame}[<+->]\frametitle{Ejemplo 2: Reducción de Bloques usando Movimiento de Bloques}
\centering
\vspace*{5mm}
\includegraphics[width=11cm]{images/example2_blockMovement2.eps}	
\end{frame}

\begin{frame}[<+->]\frametitle{Ejemplo 2: Reducción de Bloques usando Movimiento de Bloques}
\centering
\vspace*{5mm}
\includegraphics[width=11cm]{images/example2_blockMovement3.eps}	
\end{frame}

\begin{frame}[<+->]\frametitle{Ejemplo 2: Reducción de Bloques usando Movimiento de Bloques}
\centering
\vspace*{5mm}
\includegraphics[width=11cm]{images/example2_blockMovement4.eps}	
\end{frame}

\begin{frame}[<+->]\frametitle{Ejemplo 2: Reducción de Bloques usando Movimiento de Bloques}
\centering
\vspace*{5mm}
\includegraphics[width=11cm]{images/example2_blockMovement5.eps}	
\end{frame}

\section{Diagramas de Flujo}
\begin{frame}[<+->]\frametitle{Diagramas de Flujo}
\begin{columns}
\begin{column}{0.5\textwidth}
\begin{itemize}
	\item Alternativa a los diagramas de flujo.
	\item Elementos:
	\begin{itemize}
		\item Ramas: representan sistemas.
		\item Nodos: representan señales.
	\end{itemize}
	\item Las ramas tienen una flecha. Representan la dirección de flujo de la señal a través del sistema.
	\item Cada señal es igual a la suma de señales que entran al nodo.
\end{itemize}
\end{column}	
\begin{column}{0.5\textwidth}
\centering
\vspace*{5mm}
\includegraphics[width=7cm]{images/flowDiagrams.eps}	
\end{column}	
\end{columns}
\end{frame}

\begin{frame}[<+->]\frametitle{Ejemplo 3: Convertir Diagramas de Bloques en Diagramas de Flujo}
\textbf{Convierta los diagramas de bloques de las formas familiares a diagramsa de flujo.}\\
\end{frame}

\begin{frame}[<+->]\frametitle{Ejemplo 4: Convertir Diagramas de Bloques en Diagramas de Flujo}
\textbf{Convierta el siguiente diagrama de bloques en diagrama de flujo.}\\
\centering
\vspace*{5mm}
\includegraphics[width=11cm]{images/example2_blockMovement.eps}	
\end{frame}

\begin{frame}[<+->]\frametitle{Regla de Mason}
\begin{itemize}
	\item Aplicación de una fórmula obtenida por S.J. Mason (1953).
	\item Puede ser más fácil obtener la función de transferencia que usando reducción de diagramas de bloques.
	\item Definición de elementos del diagrama:
	\begin{itemize}
		\item Ganancia de lazo: Producto de ganancias de ramas que se recorren iniciando en un nodo y finalizando en el mismo nodo siguiendo la dirección de flujo, sin pasar por cualquier otro nodo más de una vez.
		\item Ganancia de trayectoria: Producto de ganancias de ramas que se recorren desde el nodo de entrada hasta el nodo de salida siguiendo la dirección de flujo.
		\item Lazos que no se tocan: Lazos que no tienen nodos en común.
	\end{itemize}
\end{itemize}
\end{frame}

\begin{frame}[<+->]\frametitle{Regla de Mason}
\begin{theorem}
\footnotesize
La función de transferencia $C(s)/R(s)$	de un sistema representado por un diagrama de flujo es:
\begin{equation*}
	G(s) = \frac{C(s)}{R(s)} = \frac{\sum_k T_k \Delta_k}{\Delta}		
\end{equation*}	
Donde:
\begin{itemize}
	\item $k$: Número de trayectorias directas.
	\item $T_k$: Ganancia de la $k$-ésima trayectoria directa.
	\item $\Delta$ = 1 - $\sum$ ganancias de lazo individuales + $\sum$ ganancias de 2 lazos que no se tocan - $\sum$ ganancias de 3 lazos que no se tocan + $\sum$ ganancias de 4 lazos que no se tocan - ...
	\item $\Delta_k$ = 1 - $\sum$ ganancias de lazo en $\Delta$ que tocan la $k$-ésima trayectoria. ($\Delta_k$ se forma eliminando de $\Delta$ las ganancias de lazo que tocan la $k$-ésima trayectoria).
\end{itemize}
\end{theorem}
\end{frame}

\begin{frame}[<+->]\frametitle{Taller}
\begin{columns}
\begin{column}{0.5\textwidth}
\centering
\vspace*{5mm} \includegraphics[width=8cm]{images/exercise1.eps}
\end{column}
\begin{column}{0.5\textwidth}
Para el diagrama de bloques de la figura:\\
\begin{itemize}
	\item Obtenga la función de transferencia usando reducción de bloques.
	\item Obtenga el diagrama de flujo equivalente.
	\item Obtenta la función de transferencia usando la regla de Mason.
	\item Verifique los resultados y concluya.
\end{itemize}
\end{column}
\end{columns}
\end{frame}


\end{document}