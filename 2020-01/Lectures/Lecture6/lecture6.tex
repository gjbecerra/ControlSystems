\documentclass[aspectratio=169,handout]{beamer}
% \usepackage[utf8]{inputenc}
\usetheme{metropolis}
\usecolortheme{orchid}
\usepackage{amsmath}
\usepackage{amssymb}
\usepackage{amsthm}
\usepackage{multirow}
\usepackage[ruled]{algorithm2e}
\usepackage{mathtools}
\usepackage{caption}
\usepackage{epstopdf}
\usepackage{hyperref}
\setbeamerfont{footnote}{size=\tiny}

\usepackage{tikz}
\usetikzlibrary{mindmap,shadows,tikzmark,positioning,arrows.meta}

% Information boxes
\newcommand*{\info}[4][16.3]{%
  \node [ annotation, #3, scale=0.65, text width = #1em,
          inner sep = 2mm ] at (#2) {%
  \list{$\bullet$}{\topsep=0pt\itemsep=0pt\parsep=0pt
    \parskip=0pt\labelwidth=8pt\leftmargin=8pt
    \itemindent=0pt\labelsep=2pt}%
    #4
  \endlist
  };
}

\tikzset{%
  >={Latex[width=2mm,length=2mm]},
  % Specifications for style of nodes:
            base/.style = {rectangle, rounded corners, draw=black,
                           minimum width=4cm, minimum height=1cm,
                           text centered, font=\sffamily},
  activityStarts/.style = {base, fill=blue!30},
       startstop/.style = {base, fill=red!30},
    activityRuns/.style = {base, fill=green!30},
         process/.style = {base, minimum width=2.5cm, fill=orange!15,
                           font=\ttfamily},
}
\renewcommand\textbullet{\ensuremath{\bullet}}
\newcommand\scalemath[2]{\scalebox{#1}{\mbox{\ensuremath{\displaystyle #2}}}}
\newcommand{\norm}[1]{\left\lVert#1\right\rVert}

%%% Bibliography
\usepackage[citestyle=numeric,style=numeric,backend=biber,doi=false,isbn=false,url=false]{biblatex}
\addbibresource{references.bib}

%%% Suppress biblatex annoying warning
\usepackage{silence}
\WarningFilter{biblatex}{Patching footnotes failed}

%%% new theorems %%%%%%%%%%%%%%%%%%%%%%%%%%%%%%%%%%%%%%%%%%%%%%%%%%%%%%%%%%%%%%
\theoremstyle{definition}
\newtheorem{mydef}{Definition}

\theoremstyle{plain}
\newtheorem{mylemma}{Lemma}[section]
\newtheorem{mytheorem}{Theorem}[section]
\newtheorem{myproposition}{Proposition}[section]
\newtheorem{myproblem}{Problem}[section]
\newtheorem{mydefinition}{Definition}[section]
\newtheorem{myassumption}{Assumption}[section]

\theoremstyle{remark}
\newtheorem{myremark}{Remark}[section]

\newcounter{saveenumi}
\newcommand{\seti}{\setcounter{saveenumi}{\value{enumi}}}
\newcommand{\conti}{\setcounter{enumi}{\value{saveenumi}}}

\resetcounteronoverlays{saveenumi}

\title{Control de Sistemas}
\subtitle{\small Clase 6: Control ON/OFF, PID}
\author{Gerardo Becerra, Ph.D.}
\institute{Pontificia Universidad Javeriana\\ Departamento de Electrónica}
\date{Febrero 25, 2020}

\begin{document}

\frame{\titlepage}	

% \frame{\tableofcontents}

% \section{Tipos de Control}
\begin{frame}[<+->]\frametitle{Clasificación Básica de Controladores}
\textbf{Se clasifican según la acción de control:}\\
\begin{enumerate}
  \item Dos posiciones (ON-OFF o BANG-BANG)
  \item Proporcional (P)
  \item Integral (I)
  \item Derivativo (D)
  \item Proporcional - Integral (PI)
  \item Proporcional - Derivativo (PD)
  \item Proporcional - Integral - Derivativo (PID)
\end{enumerate}
\end{frame}

\begin{frame}[<+->]\frametitle{Controlador ON-OFF / BANG-BANG / TODO-NADA}
\begin{columns}
  \begin{column}{0.5\textwidth}
    \vspace*{5mm}
    \includegraphics[width=7cm]{images/onoffcontroller.eps}
    \vspace*{-5mm}
    \begin{itemize}
      \item Se caracteriza porque el actuador posee sólo dos posiciones: encendido - apagado.
      \item Ventajas:
      \begin{itemize}
        \item Implementación simple.
        \item Implementación barata.
        \item Uso común en aplicaciones domésticas.
      \end{itemize}
    \end{itemize}
  \end{column} 
  \begin{column}{0.5\textwidth}
    \begin{align*}
      u(t) =
      \begin{cases}
        U_1 & \text{si } e(t) > 0\\
        U_2 & \text{si } e(t) < 0\\
      \end{cases}
    \end{align*}
    \begin{itemize}
      \item En la práctica generalmente no se usa éste controlador $\rightarrow$ puede provocar muchas conmutaciones $\rightarrow$ puede producir desgaste en el actuador.
      \item En su lugar se usa un controlador con histéresis.
    \end{itemize}
  \end{column} 
\end{columns}
\end{frame}

\begin{frame}[<+->]\frametitle{Controlador ON-OFF con Histéresis}
\begin{columns}
  \begin{column}{0.5\textwidth}
    \vspace*{5mm}
    \includegraphics[width=7cm]{images/onoffcontrollerHisteresis.eps}
    \vspace*{-5mm}
    \begin{itemize}
      \item Permite mantener el valor presente de $u(t)$ hasta que la señal de error se haya movido más allá del valor cero.
      \item La amplitud de la oscilación puede reducirse disminuyendo la brecha diferencial $\rightarrow$ más conmutaciones.
    \end{itemize}
  \end{column} 
  \begin{column}{0.5\textwidth}
    \begin{center}
    \includegraphics[width=5.5cm]{images/onoffcontrollerLevel.eps}
    \end{center}
    \vspace*{-5mm}
    \begin{itemize}
      \item No se puede utilizar para modificar transitorios.
    \end{itemize}
  \end{column} 
\end{columns}
\end{frame}

\begin{frame}[<+->]\frametitle{Controlador Proporcional (P)}
\begin{columns}
  \begin{column}{0.5\textwidth}
    \vspace*{5mm}
    \includegraphics[width=7cm]{images/Pcontroller.eps}
    \vspace*{-5mm}
    \begin{itemize}
      \item Es un amplificador de error.
      \begin{align*}
        U(s) = K_p E(s)
      \end{align*}
      \item Dependencia de $E(s)$ de la ganancia $K_p$:
      \begin{align*}
        E(s) = \frac{1}{1+K_p G(s)}R(s)
      \end{align*}
    \end{itemize}
  \end{column} 
  \begin{column}{0.5\textwidth}
    \begin{itemize}
      \item $K_p$ aumenta $\Rightarrow$ menor error de estado estacionario.
      \item $K_p$ muy grande no elimina el error de estado estacionario:
      \item $K_p$ muy grande puede ocasionar inestabilidad.
    \end{itemize}
  \end{column} 
\end{columns}
\end{frame}

\begin{frame}[<+->]\frametitle{Controlador Proporcional (P) - Error de Estado Estacionario}
  \begin{align*}
    E(s) = \frac{1}{1+K_p G(s)}R(s)
  \end{align*}
  \pause
  Asumiendo $G(s)$ de la forma:
  \begin{align*}
    G(s) = \frac{a_n s^n + a_{n-1} s^{n-1} + \dots + a_1 s + a_0}{b_m s^m + b_{m-1} s^{m-1} + \dots + b_1 s + b_0}
  \end{align*}
  \pause
  Usando el teorema del valor final: 
  \begin{align*}
    e_{ss} = \lim_{s \rightarrow 0} s \left(\frac{1}{s}\right) \left(\frac{1}{1 + K_p G(s)}\right) = \frac{1}{1 + K_p \frac{a_0}{b_0}}
  \end{align*}
\end{frame}

\begin{frame}[<+->]\frametitle{Controlador Proporcional (P) - Ejemplo}
  Suponga  $G(s) = \frac{1}{s+a}$. Entonces la ganancia de lazo cerrado será:
  \begin{align*}
    G_{lc}(s) = \frac{K_p}{s+a+K_p}
  \end{align*}
  \pause
  Si $R(s) = 1/s$ (escalón unitario) entonces:
  \begin{align*}
    Y(s) = \left(\frac{K_p}{s+a+K_p}\right) \frac{1}{s}
  \end{align*}
  \pause
  Aplicando la transformada inversa de Laplace:
  \begin{equation*}
    y(t) = \frac{K_p}{K_p+a} \left(1-e^{-t(K_p+a)}\right)
  \end{equation*}
  \pause
  Si $K_p$ se incrementa, el sistema será más rápido (no es el caso general).
\end{frame}

\begin{frame}[<+->]\frametitle{Controlador Proporcional (P) - Rechazo a Perturbaciones}
\begin{columns}
\begin{column}{0.5\textwidth}
\vspace*{-3mm}
\begin{center}
  \includegraphics[width=7.5cm]{images/PcontrollerwithDisturbance.eps}\\
\end{center}
\pause
Asumiendo $R(s) = 0$:
\begin{equation*}
  \frac{Y(s)}{D(s)} = \frac{G(s)}{1 + K_p G(s)}
\end{equation*}
\pause
Asumiendo $G(s)$ de la forma:
\begin{align*}
  G(s) = \frac{a_n s^n + a_{n-1} s^{n-1} + \dots + a_1 s + a_0}{b_m s^m + b_{m-1} s^{m-1} + \dots + b_1 s + b_0}
\end{align*}
\end{column}
\pause
\begin{column}{0.5\textwidth}
  Usando el teorema del valor final, asumiendo perturbación constante: 
  \begin{align*}
    y_{ss} &= \lim_{s \rightarrow 0} s \left(\frac{1}{s}\right) \left(\frac{G(s)}{1 + K_p G(s)}\right)\\
    &= \frac{a_0}{b_0 + K_p a_0}
  \end{align*}
  \pause
  Incrementar $K_p$ reduce el efecto de la perturbación, pero no lo elimina. No es inmune al ruido.
\end{column}
\end{columns}
\end{frame}

\begin{frame}[<+->]\frametitle{Controlador Proporcional (P) - Ventajas y Desventajas}
  \begin{itemize}
    \item Ventajas:
    \begin{itemize}
      \item Instantaneidad de aplicación.
      \item Facilidad para comprobar resultados.
    \end{itemize}
    \item Desventajas:
    \begin{itemize}
      \item Falta de inmunidad a ruido.
      \item No corrige algunos errores en estado estacionario.
      \item Puede hacer inestable al sistema.
    \end{itemize}
  \end{itemize}
\end{frame}

\begin{frame}[<+->]\frametitle{Controlador Integral (I)}
% \vspace*{-5mm}
\begin{columns}
\begin{column}{0.5\textwidth}
\begin{center}
  \includegraphics[width=7cm]{images/Icontroller.eps}
\end{center}
\pause
\begin{equation*}
  u(t) = \int_0^t e(t) dt \longrightarrow \frac{U(s)}{E(s)} = \frac{K_i}{s}
\end{equation*}
\pause
Ganancia de lazo cerrado:
\begin{equation*}
  G_{lc}(s) = \frac{Y(s)}{R(s)} = \frac{K_i G(s)}{s + K_i G(s)}
\end{equation*}
\pause
Dependencia de $E(s)$ de la ganancia $K_i$:
\begin{equation*}
E(s) = (1-G_{lc}(s)) = \frac{s}{s+K_i G(s)} R(s)
\end{equation*}
\pause
\end{column}
\begin{column}{0.5\textwidth}
Asumiendo $G(s)$ de la forma:
\begin{align*}
  G(s) = \frac{a_n s^n + a_{n-1} s^{n-1} + \dots + a_1 s + a_0}{b_m s^m + b_{m-1} s^{m-1} + \dots + b_1 s + b_0}
\end{align*}
\pause
Usando el teorema del valor final:
\begin{equation*}
  e_{ss} = \lim_{s \rightarrow 0} s \frac{s}{s + K_i \frac{a_n s^n + a_{n-1} s^{n-1} + \dots + a_1 s + a_0}{b_m s^m + b_{m-1} s^{m-1} + \dots + b_1 s + b_0}}R(s) 
\end{equation*}
\pause
Si $R(s) = 1/s$, entonces $e_{ss} = 0$.\\
\textbf{El integrador anula el error de estado estable.}
\end{column}
\end{columns}
\end{frame}

\begin{frame}[<+->]\frametitle{Controlador Integral (I) - Rechazo a Perturbaciones}
\begin{align*}
  \frac{Y(s)}{D(s)} &= \frac{G(s)}{1+\frac{K_i}{s}G(s)} = \frac{\frac{a_n s^n + \dots + a_0}{b_m s^m + \dots + b_0}}{1 + \frac{K_i}{s}\frac{a_n s^n + \dots + a_0}{b_m s^m + \dots + b_0}}\\
  &= \frac{(a_n s^n + \dots + a_0)s}{(b_m s^m + \dots + b_0)s + K_i(a_n s^n + \dots + a_0)} 
\end{align*}
\pause
Si $D(s) = 1/s$ (escalón unitario), entonces
\begin{equation*}
  y_{ss} = \lim_{s \rightarrow 0} s \frac{1}{s} \frac{(a_n s^n + \dots + a_0)s}{(b_m s^m + \dots + b_0)s + K_i(a_n s^n + \dots + a_0)} = 0
\end{equation*}
\pause
\textbf{El integrador elimina perturbaciones en estado estacionario.}
\end{frame}

\begin{frame}[<+->]\frametitle{Controlador Integral (I) - Ventajas y Desventajas}
  \begin{itemize}
    \item Ventajas:
    \begin{itemize}
      \item Elimina error de estado estacionario.
      \item Elimina efecto de las perturbaciones (robustez).
    \end{itemize}
    \item Desventajas:
    \begin{itemize}
      \item La aplicación no es instantánea.
      \item Puede agregar inestabilidad al sistema total debido al polo en el origen.
    \end{itemize}
  \end{itemize}
\end{frame}

\begin{frame}[<+->]\frametitle{Controlador Proporcional - Integral (PI)}
\begin{center}
  \includegraphics[width=8cm]{images/PIcontroller.eps}
\end{center}
\pause
\begin{equation*}
  u(t) = K_p e(t) + K_i \int_0^t e(t) dt \longrightarrow U(s) = K_p E(s) + K_i \frac{E(s)}{s}
\end{equation*}
\pause
\begin{equation*}
  \frac{U(s)}{E(s)} = K_p + \frac{K_i}{s} = \frac{K_p s + K_i}{s}
\end{equation*}
% \pause
% \begin{equation*}
%   \frac{U(s)}{E(s)} = K_p \left(1 + \frac{1}{T_i s} \right),\ T_i\text{: Tiempo integral}
% \end{equation*}
\end{frame}

\begin{frame}[<+->]\frametitle{Controlador Proporcional - Integral (PI)}
\begin{align*}
  G_{lc}(s) = \frac{Y(s)}{R(s)} = \frac{\frac{K_ps + K_i}{s}G(s)}{1 + \frac{K_ps + K_i}{s}G(s)} = \frac{(K_ps + K_i)G(s)}{s + (K_ps+K_i)G(s)}
\end{align*}
\pause
\begin{align*}
  E(s) = R(s) - Y(s) = R(s) - G_{lc}(s)R(s)
\end{align*}
\pause
\begin{align*}
  \frac{E(s)}{R(s)} = 1 - G_{lc}(s) = 1 - \frac{(K_ps + K_i)G(s)}{s + (K_ps+K_i)G(s)} = \frac{s}{s + (K_ps + K_i)G(s)}
\end{align*}
\pause
Aplicando el teorema del valor final:
\begin{equation*}
  e_{ss} = \lim_{s \rightarrow 0} s \frac{s}{s + (K_ps + K_i)G(s)} R(s)
\end{equation*}
\pause
Si $R(s) = 1/s$ (escalón unitario), entonces $e_{ss} = 0$. \textbf{El controlador anula el error de estado estacionario.}
\end{frame}

\begin{frame}[<+->]\frametitle{Controlador Proporcional - Integral (PI) - Rechazo a Perturbaciones}
\begin{equation*}
  \frac{Y(s)}{D(s)} = \frac{G(s)}{1 + G(s)\frac{K_ps + K_i}{s}} = \frac{G(s)s}{s + G(s)(K_ps+K_i)}
\end{equation*}
\pause
Si $D(s) = 1/s$, entonces:
\begin{equation*}
  y_{ss} = \lim_{s \rightarrow 0} s \frac{1}{s} \frac{G(s)s}{s + G(s)(K_ps+K_i)} = 0
\end{equation*}
\pause
\textbf{El controlador elimina el efecto de las perturbaciones.}
\end{frame}

\begin{frame}[<+->]\frametitle{Controlador Proporcional - Integral (PI) - Ventajas}
\begin{itemize}
  \item Respuesta inmediata
  \item Elimina error de estado estacionario.
  \item Elimina efecto de las perturbaciones (robustez).
\end{itemize}
\end{frame}

\begin{frame}[<+->]\frametitle{Control Derivativo (D)}
\begin{center}
  \includegraphics[width=8cm]{images/Dcontroller.eps}
\end{center}
\pause
\begin{equation*}
  u(t) = K_d \frac{de(t)}{dt} \longrightarrow U(s) = K_d s E(s) \rightarrow \frac{U(s)}{E(s)} = K_d s
\end{equation*}
\pause
\begin{align*}
  \frac{E(s)}{R(s)} = 1 - G_{lc}(s) = 1 - \frac{G(s) K_d s}{1 + G(s) K_d s} = \frac{1}{1 + K_d s G(s)}
\end{align*}
\pause
\begin{equation*}
  e_{ss} = \lim_{s \rightarrow 0} s \frac{A}{s} \frac{1}{1 + K_d s G(s)} = A
\end{equation*}
\pause
\textbf{No elimina el error de estado estacionario!}
\end{frame}

\begin{frame}[<+->]\frametitle{Control Derivativo (D) - Rechazo a Perturbaciones}
Si $R(s) = 0$:
\begin{equation*}
  \frac{Y(s)}{D(s)} = \frac{G(s)}{1 + G(s) K_d s}
\end{equation*}
\pause
Asumiendo $D(s) = A/s$:
\begin{equation*}
  y_{ss} = \lim_{s \rightarrow 0} s \frac{A_p}{s} \frac{G(s)}{1 + G(s) K_d s} = A_p G(s)
\end{equation*}
\pause
\textbf{No elimina perturbaciones!}
\end{frame}

\begin{frame}[<+->]\frametitle{Control Derivativo (D) - Análisis en Tiempo}
\begin{columns}
\begin{column}{0.3\textwidth}
\begin{center}
  \includegraphics[width=5cm]{images/Dcontroller_time.eps}
\end{center}
\end{column} 
\begin{column}{0.7\textwidth}
  \begin{itemize}
    \item Debido a la dinámica del proceso, pasa algún tiempo hasta que cambios en la variable de control produzcan cambios en la salida del proceso.
    \item La acción derivativa trata de \textit{predecir} o \textit{anticipar} el valor que tomará la señal de error para dar un valor a la señal de control.
    \item Ventaja: Es anticipativo. adelanta la acción de control frente a la aparición de una tendencia del error (sólo aplica a transitorios).
    \item Desventaja: Inaplicable ante la presencia de ruido.
  \end{itemize}
\end{column} 
\end{columns}
\end{frame}

\begin{frame}[<+->]\frametitle{Control Proporcional - Derivativo (PD)}
\begin{center}
  \includegraphics[width=8cm]{images/PDcontroller.eps}
\end{center}
\pause
\begin{equation*}
  u(t) = K_p e(t) + K_d \frac{de(t)}{dt} \longrightarrow U(s) = K_p E(s) + K_d s E(s) \rightarrow \frac{U(s)}{E(s)} = K_p + K_d s
\end{equation*}
\begin{columns}
\begin{column}{0.5\textwidth}
Ventajas:
\begin{itemize}
  \item Reduce el error de estado estable.
  \item Adelanta la acción de control (es anticipativo).
  \item La respuesta es inmediata.
\end{itemize}
\end{column}
\begin{column}{0.5\textwidth}
Desventajas:
\begin{itemize}
  \item No corrije errores en estado estacionario.
  \item Es sensible a ruidos. Generalmente se utiliza con un filtro en la entrada.
\end{itemize}
\end{column}
\end{columns}
\end{frame}

\begin{frame}[<+->]\frametitle{Control Proporcional - Integral - Derivativo (PID)}
\begin{center}
  \includegraphics[width=8cm]{images/PIDcontroller.eps}
\end{center}
\pause
\begin{align*}
  u(t) = K_p e(t) + K_i \int_0^t e(t) dt + K_d \frac{de(t)}{dt}
\end{align*}
\pause
\begin{align*}
  U(s) = K_p E(s) + K_i \frac{E(s)}{s} + K_d s E(s) \rightarrow \frac{U(s)}{E(s)} = K_p + \frac{K_i}{s} + K_d s
\end{align*}
\end{frame}

\begin{frame}[<+->]\frametitle{Control Proporcional - Integral - Derivativo (PID)}
\begin{center}
  \includegraphics[width=8cm]{images/PIDcontroller.eps}
\end{center}
\begin{columns}
\begin{column}{0.5\textwidth}
Ventajas:
\begin{itemize}
  \item Elimina error de estado estacionario (I).
  \item Elimina efecto de las perturbaciones (I).
  \item Es anticipativo (D).
  \item Tiene una acción inmediata (P).
\end{itemize}
\end{column}
\begin{column}{0.5\textwidth}
Desventajas:
\begin{itemize}
  \item Tiene un polo en el origen (genera inestabilidad).
  \item La parte derivativa amplifica el ruido.
\end{itemize}
\end{column}
\end{columns}
\end{frame}

\begin{frame}[<+->]\frametitle{Sintonización de Controladores}
\textbf{Cómo encontrar los valores $K_p$, $K_i$, $K_d$?} No existe una manera única de encontrarlos! Existen muchos métodos:
\begin{itemize}
  \item Experimental.
  \item Análisis de una F.T.
  \item Técnicas de optimización.
  \item Lugar geométrico de las raíces.
  \item Compensación en frecuencia.
\end{itemize}
\end{frame}

\begin{frame}[c]\frametitle{Ejemplo: Control PID de un Sistema de Segundo Orden}
\footnotesize
Considere el siguiente lazo de control:
\begin{center}
  \includegraphics[width=8cm]{images/example2.eps}
\end{center}
\vspace*{-5mm}
\begin{enumerate}
  \item Calcule el valor de la ganancia $K_p$ para obtener un sobrepico de 10\% ante una entrada paso unitaria. Realice la simulación y verifique el resultado.
  \item Considere ahora el sistema $G(s) = \frac{1}{s^2 + 3.6s + 1}$. Cómo cambia la respuesta al simular el sistema con la ganancia encontrada anteriormente?
  \item Usando prueba y error, encuentre ganancias $K_p$ y $K_i$ de un control PI que mejore la respuesta del nuevo sistema.
  \item Usando prueba y error, encuentre ganancias $K_p$ y $K_d$ de un control PD que mejore la respuesta del nuevo sistema.
  \item Usando prueba y error, encuentre ganancias $K_p$, $K_i$ y $K_d$ de un control PID que mejore la respuesta del nuevo sistema.
\end{enumerate}
\end{frame}

\end{document}