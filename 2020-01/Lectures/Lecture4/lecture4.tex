\documentclass[aspectratio=169,handout]{beamer}
% \usepackage[utf8]{inputenc}
\usetheme{metropolis}
\usecolortheme{orchid}
\usepackage{amsmath}
\usepackage{amssymb}
\usepackage{amsthm}
\usepackage{multirow}
\usepackage[ruled]{algorithm2e}
\usepackage{mathtools}
\usepackage{caption}
\usepackage{epstopdf}
\usepackage{hyperref}
\usepackage{verbatim}
\usepackage[makeroom]{cancel}
\setbeamerfont{footnote}{size=\tiny}
\setbeamertemplate{theorems}[ams style] 

\usepackage{tikz}
\usetikzlibrary{mindmap,shadows,tikzmark,positioning,arrows.meta}

% Information boxes
\newcommand*{\info}[4][16.3]{%
  \node [ annotation, #3, scale=0.65, text width = #1em,
          inner sep = 2mm ] at (#2) {%
  \list{$\bullet$}{\topsep=0pt\itemsep=0pt\parsep=0pt
    \parskip=0pt\labelwidth=8pt\leftmargin=8pt
    \itemindent=0pt\labelsep=2pt}%
    #4
  \endlist
  };
}

\tikzset{%
  >={Latex[width=2mm,length=2mm]},
  % Specifications for style of nodes:
            base/.style = {rectangle, rounded corners, draw=black,
                           minimum width=4cm, minimum height=1cm,
                           text centered, font=\sffamily},
  activityStarts/.style = {base, fill=blue!30},
       startstop/.style = {base, fill=red!30},
    activityRuns/.style = {base, fill=green!30},
         process/.style = {base, minimum width=2.5cm, fill=orange!15,
                           font=\ttfamily},
}
\renewcommand\textbullet{\ensuremath{\bullet}}
\newcommand\scalemath[2]{\scalebox{#1}{\mbox{\ensuremath{\displaystyle #2}}}}
\newcommand{\norm}[1]{\left\lVert#1\right\rVert}

%%% Bibliography
\usepackage[citestyle=numeric,style=numeric,backend=biber,doi=false,isbn=false,url=false]{biblatex}
\addbibresource{references.bib}

%%% Suppress biblatex annoying warning
\usepackage{silence}
\WarningFilter{biblatex}{Patching footnotes failed}

%%% new theorems %%%%%%%%%%%%%%%%%%%%%%%%%%%%%%%%%%%%%%%%%%%%%%%%%%%%%%%%%%%%%%
\theoremstyle{definition}
\newtheorem{mydef}{Definition}

\theoremstyle{plain}
\newtheorem{mylemma}{Lemma}[section]
\newtheorem{mytheorem}{Theorem}[section]
\newtheorem{myproposition}{Proposition}[section]
\newtheorem{myproblem}{Problem}[section]
\newtheorem{mydefinition}{Definition}[section]
\newtheorem{myassumption}{Assumption}[section]

\theoremstyle{remark}
\newtheorem{myremark}{Remark}[section]

\newcounter{saveenumi}
\newcommand{\seti}{\setcounter{saveenumi}{\value{enumi}}}
\newcommand{\conti}{\setcounter{enumi}{\value{saveenumi}}}

\resetcounteronoverlays{saveenumi}

\title{Control de Sistemas}
\subtitle{\small Clase 4: Estabilidad de Sistemas Retroalimentados}
\author{Gerardo Becerra, Ph.D.}
\institute{Pontificia Universidad Javeriana\\ Departamento de Electrónica}
\date{Febrero 18, 2020}

\begin{document}

\frame{\titlepage}	

% \frame{\tableofcontents}

\begin{frame}[<+->]\frametitle{Introducción}
\begin{itemize}
	\item Estabilidad $\rightarrow$ muy importante para el diseño y análisis de sistemas retroalimentados.
	\item Sistema de lazo cerrado inestable $\rightarrow$ en general no es muy útil.
	\item Sistemas inestables $\rightarrow$ pueden hacerse estables usando retroalimentación.
	\item Sistemas estables $\rightarrow$ pueden mejorar su desempeño usando retroalimentación.
	\item Tipos de estabilidad:
	\begin{itemize}
		\item Estabilidad externa: asociada con las entradas.
		\item Estabilidad interna: asociada al estado interno.
	\end{itemize}
\end{itemize}
\end{frame}

\section{Estabilidad Externa}
\begin{frame}[<+->]\frametitle{Estabilidad Externa}
\begin{itemize}
 	\item Sistema SISO (single input, single output), lineal e invariante en el tiempo, con respuesta impulso $g(t)$:
	\begin{equation*}
		y(t) = \int_0^t g(t-\tau) u(\tau) d\tau
	\end{equation*}
	\item Se asume sistema causal y relajado en $t=0$.
	\item Entrada $u(t)$ acotada (bounded): no crece indefinidamente. Existe una constante $u_m$ tal que $\norm{u(t)} \leq u_m \leq \infty$ para todo $t >= 0$.
  \item \textbf{Estabilidad BIBO} (bounded input - bounded output): toda entrada acotada produce una salida acotada.
 \end{itemize} 
\end{frame}

\begin{frame}[<-+>]\frametitle{Estabilidad Externa}
  \begin{theorem}\label{th:externalStability1}
  Un sistema es BIBO estable si y sólo si $g(t)$ es absolutamente integrable en el intervalo $[0,\infty)$ para alguna constante $M$:
  \begin{equation*}
  	\int_0^\infty |g(t)| dt \leq M \leq \infty
  \end{equation*}
  \end{theorem}
\end{frame}

\begin{frame}[<-+>]\frametitle{Estabilidad Externa}
	Para sistemas definidos en términos de función de transferencia:
  \begin{theorem}\label{th:externalStability2}
    Un sistema con función de transferencia racional propia $G(s)$ es BIBO estable si y sólo si todos los polos de $G(s)$ tienen parte real negativa, o se encuentran en el lado izquierdo del semiplano complejo, sin incluir el eje imaginario.
  \end{theorem}
\end{frame}

\begin{frame}[<+->][fragile]\frametitle{Estabilidad Externa: Ejemplo}
\textbf{Considere el sistema dado por la ecuación diferencial $\ddot{y} + \dot{y} - 2y = \dot{u} - u$. Verifique si el sistema es BIBO estable.}\\
\small \pause
\begin{columns}
\begin{column}{0.5\textwidth}
Aplicando la transformada de Laplace, se obtiene la función de transferencia:
\begin{equation*}
	G(s) = \frac{Y(s)}{U(s)} = \frac{s-1}{s^2+s-2}
\end{equation*}
\pause
\end{column}	
\begin{column}{0.5\textwidth}
Usando Matlab para graficar la respuesta impulso:
\begin{verbatim}
>> G = tf([1 -1],[1 1 -2])
impulse(G)
G =
     s - 1
  -----------
  s^2 + s - 2
 
Continuous-time transfer function.
\end{verbatim}
\end{column}	
\end{columns}
\end{frame}

\begin{frame}[<+->][fragile]\frametitle{Estabilidad Externa: Ejemplo}
\centering
\includegraphics[width=9cm]{images/impulseResponse.eps}\\
\small \pause
Teorema \ref{th:externalStability1}: $g(t)$ absolutamente integrable $\Rightarrow$ BIBO estable.
\end{frame}

\begin{frame}[<+->][fragile]\frametitle{Estabilidad Externa: Ejemplo}
Calculando los polos de la función de transferencia:
\begin{verbatim}
>> pole(G)

ans =

    -2
     1
\end{verbatim}
\pause
\textcolor{red}{No todos los polos tienen parte real negativa! Cuál es el problema?}
\end{frame}

\begin{frame}[<+->][fragile]\frametitle{Estabilidad Externa: Ejemplo}
\vspace*{5mm}
\begin{columns}
\begin{column}{0.5\textwidth}
Cancelación polo - cero
\begin{equation*}
	G(s) = \frac{Y(s)}{U(s)} = \frac{\cancel{s-1}}{(s+2)\cancel{(s-1)}} = \frac{1}{s+2}
\end{equation*}
\pause
Entonces, el sistema satisface las condiciones del teorema \ref{th:externalStability2} $\Rightarrow$ es BIBO estable.
\end{column}	
\begin{column}{0.5\textwidth}
Aplicando una entrada paso:
\centering
\includegraphics[width=7cm]{images/stepResponse.eps}
Respuesta acotada ante una entrada acotada.
\end{column}	
\end{columns}
\end{frame}

\begin{frame}[<+->]\frametitle{Criterio de Estabilidad de Routh-Hurwitz}
\begin{itemize}
	\item A. Hurwitz [1895] y E. J. Routh [1892]: método para investigar la estabilidad de un sistema lineal.
	\item Provee una respuesta a la pregunta por la estabilidad considerando el polinomio característico del sistema:
	\begin{equation*}
		q(s) = a_n s^n + a_{n-1} s^{n-1} + \dots + a_1 s + a_0 = 0
	\end{equation*}
	sin necesidad de calcular las raíces.
	\item Recientemente no es tan usado gracias a los métodos numéricos para calcular las raíces.
	\item Factorizando $q(s)$ se obtiene:
	\begin{equation*}
		a_n (s-r_1) (s-r_1) \dots (s-r_n) = 0
	\end{equation*}
	donde $r_i$ corresponde a la $i$-esima raíz de $q(s)$.
\end{itemize}
\end{frame}

\begin{frame}[<+->]\frametitle{Criterio de Estabilidad de Routh-Hurwitz}
\begin{itemize}
	\item Multiplicando los factores se obtiene:
	\begin{align*}
		q(s) = a_n& s^n - a_n(r_1 + r_2 + \dots + r_n)s^{n-1} + a_n(r_1r_2 + r_2r_3 + \dots)s^{n-2}\\
		&- a_n(r_1r_2r_3 + r_1r_2r_4 + \dots)s^{n-3} + \dots - a_n(-1)^n(r_1r_2r_3 \dots r_n) = 0
	\end{align*}
	\item Entonces se obtiene la forma general:
	\begin{align*}
		q(s) = a_n& s^n - a_n(\text{suma de todas las raíces})s^{n-1}\\
		&+ a_n(\text{suma de productos de combinaciones de 2 raíces})s^{n-2}\\
		&- a_n(\text{suma de productos de combinaciones de 3 raíces})s^{n-3} + \dots \\
		&- a_n(-1)^n(\text{productos de todas las raíces}) = 0
	\end{align*}
	\item Si las raíces están en el semiplano izquierdo, los coeficientes tienen el mismo signo y son diferentes a cero (condición necesaria pero no suficiente).
\end{itemize}
\end{frame}

\begin{frame}[<+->]\frametitle{Criterio de Estabilidad de Routh-Hurwitz}
\begin{itemize}
	\item Criterio de Routh-Hurwitz: condición necesaria y suficiente para estabilidad de sistemas lineales.
	\item A partir del polinomio característico $a_n s^n + a_{n-1} s^{n-1} + \dots + a_1 s + a_0 = 0$, se define el arreglo de Routh como:
	\vspace{-5mm}
	\begin{columns}
	\begin{column}{0.5\textwidth}
	\begin{tabular}{c|cccc}
		$s^n$     & $a_n$     & $a_{n-2}$ & $a_{n-4}$ & \dots\\
		$s^{n-1}$ & $a_{n-1}$ & $a_{n-3}$ & $a_{n-5}$ & \dots\\
		$s^{n-2}$ & $b_{n-1}$ & $b_{n-3}$ & $b_{n-5}$ & \dots\\
		$s^{n-3}$ & $c_{n-1}$ & $c_{n-3}$ & $c_{n-5}$ & \dots\\
		$\vdots$ & $\vdots$ & $\vdots$ & $\vdots$ & $\ddots$\\
		$s^0$ & $h_{n-1}$ & & & 
	\end{tabular}
	\end{column}
	\begin{column}{0.5\textwidth}
	\begin{align*}
		b_{n-1} &= \frac{-1}{a_{n-1}}
		\begin{vmatrix}
			a_n     & a_{n-2}\\
			a_{n-1} & a_{n-3}
		\end{vmatrix},\\
		b_{n-3} &= \frac{-1}{a_{n-1}}
		\begin{vmatrix}
			a_n     & a_{n-4}\\
			a_{n-1} & a_{n-5}
		\end{vmatrix}, \dots\\
		c_{n-1} &= \frac{-1}{b_{n-1}}
		\begin{vmatrix}
			a_{n-1} & a_{n-3}\\
			b_{n-1} & b_{n-3}
		\end{vmatrix}, \dots\\
	\end{align*}
	\end{column}
	\end{columns}
	\vspace{-5mm}
	\item Número de raíces de $q(s)$ con parte real positiva = número de cambios de signo en la primera columna del arreglo de Routh.
\end{itemize}
\end{frame}

\begin{frame}[<+->]\frametitle{Criterio de Estabilidad de Routh-Hurwitz - Caso 1}
\textbf{Caso 1: Ningún elemento de la primera columna es cero.}
\begin{itemize}
	\item Polinomio característico: $q(s) = a_2 s^2 + a_1 s + a_0 = 0$.
	\item Arreglo de Routh:\\
	\centering	
	\begin{columns}
	\begin{column}{0.2\textwidth}
	\begin{tabular}{c|cc}
		$s^2$ & $a_2$ & $a_0$\\
		$s^1$ & $a_1$ & 0\\
		$s^0$ & $b_1$ & 0
	\end{tabular}
	\end{column}	
	\begin{column}{0.2\textwidth}
	\begin{equation*}
		b_1 = \frac{-1}{a_1}
		\begin{vmatrix}
			a_2 & a_0\\
			a_1 & a_0
		\end{vmatrix} = a_0
	\end{equation*}
	\end{column}	
	\end{columns}
	\item Entonces, el sistema es estable si todos los coeficientes son positivos o todos son negativos.
\end{itemize}
\end{frame}

\begin{frame}[<+->]\frametitle{Criterio de Estabilidad de Routh-Hurwitz - Caso 2}
\textbf{Caso 2: Existe un cero en la primera columna, pero otros elementos de esa fila son diferentes a cero.}
\begin{itemize}
	\item Polinomio característico: $q(s) = s^5 + 2s^4 + 2s^3 + 4s^2 + 11s + 10 = 0$.
	\item Arreglo de Routh:\\
	\centering	
	\begin{columns}
	\begin{column}{0.2\textwidth}
	\begin{tabular}{c|ccc}
		$s^5$ & 1 & 2 & 11\\
		$s^4$ & 2 & 4 & 10\\
		$s^3$ & $\epsilon$ & 6 & 0\\
		$s^2$ & $c_1$ & 10 & 0\\
		$s^1$ & $d_1$ & 0 & 0\\
		$s^0$ & 10 & 0 & 0
	\end{tabular}
	\end{column}	
	\begin{column}{0.2\textwidth}
	\begin{align*}
		c_1 &= \frac{4\epsilon - 12}{\epsilon}\\
		d_1 &= \frac{6c_1 - 10\epsilon}{c_1}
	\end{align*}
	\end{column}	
	\end{columns}
	\item Cuando $\epsilon \rightarrow 0^+$, se obtiene $c_1 < 0$ y $d_1 > 0$.
	\item Entonces, hay dos cambios de signo en la primera columna.
	\item Por lo tanto, el sistema es inestable con dos raíces en el semiplano derecho.
\end{itemize}
\end{frame}

\begin{frame}[<+->]\frametitle{Criterio de Estabilidad de Routh-Hurwitz - Caso 3}
\textbf{Caso 3: Todos los coeficientes en una fila son cero.}
\begin{columns}
\begin{column}{0.5\textwidth}
	\begin{itemize}
		\item Polinomio característico: $q(s) = s^5 + 2s^4 + 24s^3 + 48s^2 - 25s - 50 = 0$.
		\item Arreglo de Routh:\\
		\centering	
		\begin{tabular}{c|ccc}
			$s^5$ & 1 & 24 & -25\\
			$s^4$ & 2 & 48 & -50\\
			$s^3$ & 0 & 0  & 
		\end{tabular}
		\item Se forma el polinomio auxiliar usando la fila anterior: $P(s) = 2s^4 + 48s^2 - 50$.
	\end{itemize}	
\end{column}	
\begin{column}{0.5\textwidth}
	\begin{itemize}
		\item La fila se reemplaza por los coeficientes de $dP(s)/ds = 8s^3 + 96s$:
		\centering	
		\begin{tabular}{c|ccc}
			$s^5$ & 1  &  24 & -25\\
			$s^4$ & 2  &  48 & -50\\
			$s^0$ & 8  &  96 & \\
			$s^2$ & 24 & -50 & \\
			$s^1$ & 112.7 & 0 & \\
			$s^0$ & -50 &  & 
		\end{tabular}
		\item Hay un cambio de signo $\rightarrow$ una raíz con parte real positiva $\rightarrow$ sistema inestable.
	\end{itemize}
\end{column}	
\end{columns}
\end{frame}

\section{Estabilidad Interna}
\begin{frame}[<+->]\frametitle{Estabilidad Interna}
\begin{itemize}
  \item Se estudia para el caso de respuesta a entrada cero: $\dot{x}(t) = \mathbf{A}x(t)$
  \item Se desea conocer si el sistema es internamente estable para una condición inicial $x_0$.
  \item Para el siguiente sistema, qué podría decirse sobre la estabilidad en cada caso?
\end{itemize}
	\centering
	\includegraphics[width=8cm]{images/internalStability.eps}
\end{frame}

\begin{frame}[<+->]\frametitle{Estabilidad Interna}
\vspace*{5mm}
\begin{columns}
\begin{column}{0.5\textwidth}
Se presentan dos criterios de estabilidad:
\begin{itemize}
  \item Estable en el sentido de Lyapunov: toda condición inicial limitada genera una respuesta limitada del estado.
  \item Asintóticamente estable: toda condición inicial limitada genera una respuesta del estado limitada que tiende a cero cuando el tiempo tiende a infinito. 
\end{itemize}
\end{column}	
\begin{column}{0.5\textwidth}
\centering
\includegraphics[width=7cm]{images/stabilityLyapunov.png}
\end{column}	
\end{columns}
\end{frame}

\begin{frame}[<-+>]\frametitle{Estabilidad Interna}
\begin{theorem}\label{th:internalStability1}
\begin{itemize}
  \item El sistema es asintóticamente estable si y sólo si todos los valores propios de $\mathbf{A}$ tienen parte real negativa.
  \item El sistema es marginalmente estable si y sólo si todos los valores propios de $\mathbf{A}$ tienen parte real cero o negativa, y aquellos con parte real cero son raices simples del polinomio característico de $\mathbf{A}$.
\end{itemize}
\end{theorem}
\end{frame}

\begin{frame}[<+->][fragile]\frametitle{Estabilidad Interna - Ejemplo}
El sistema definido por la ecuación diferencial $\ddot{y} + \dot{y} - 2y = \dot{u} - u$ puede escribirse en variables de estado como:
\begin{columns}
\begin{column}{0.5\textwidth}
\scriptsize
\begin{verbatim}
>> [A,B,C,D] = tf2ss([1 -1],[1 1 -2])
A =
    -1     2
     1     0
B =
     1
     0
C =
     1    -1
D =
     0
\end{verbatim} 
\end{column}	
\begin{column}{0.5\textwidth}
\begin{align*}
\begin{bmatrix}
	\dot{x}_1 \\ \dot{x}_2
\end{bmatrix} = 
\begin{bmatrix}
	-1 & 2 \\ 1 & 0
\end{bmatrix}
\begin{bmatrix}
	x_1 \\ x_2
\end{bmatrix} + 
\begin{bmatrix}
	1 \\ 0
\end{bmatrix}u\\
y =
\begin{bmatrix}
	1 & -1
\end{bmatrix}
\begin{bmatrix}
	x_1 \\ x_2
\end{bmatrix} + 
\begin{bmatrix}
	0
\end{bmatrix}u\\
\end{align*}
\vspace*{-1.5cm}
\small
\begin{verbatim}
>> eig(A)
ans =
    -2
     1	
\end{verbatim}
\textbf{Teorema \ref{th:internalStability1} $\rightarrow$ internamente inestable!}
\end{column}	
\end{columns}
\end{frame}

\begin{frame}[<+->][fragile]\frametitle{Estabilidad Interna - Ejemplo}
\vspace*{5mm}
\begin{columns}
\begin{column}{0.4\textwidth}
\small
\begin{verbatim}
sys = ss(A,B,C,D);
[y,t,x] = initial(sys,[1,0]);
\end{verbatim}
\textcolor{red}{Estados inestables, pero salida estable!} 
\end{column}	
\begin{column}{0.6\textwidth}
\centering
\includegraphics[width=9cm]{images/initialCondResponse.eps}
\end{column}	
\end{columns}
\end{frame}

\begin{frame}[<-+>]\frametitle{Relación entre los Diferentes Tipos de Estabilidad}
\centering
\vspace*{5mm}
\includegraphics[width=8.5cm]{images/stabilityTypes.eps}
\end{frame}

\begin{frame}\frametitle{Taller}
\begin{enumerate}
  \conti
  \item El sistema mostrado en la figura representa un proceso que controlador por un controlador proporcional-derivativo (PD).
  \begin{itemize}
  	\item Usando el criterio de Routh-Hurwitz, determine el rango de $K_P$ y $K_D$ para garantizar estabilidad del sistema en lazo cerrado.
  	\item Asigne valores a $K_P$ y $K_D$ dentro del rango encontrado. Verifique para éste caso las propiedades de estabilidad interna y externa usando los teoremas \ref{th:externalStability2} y \ref{th:internalStability1}.
  \end{itemize}
  \seti
\end{enumerate}
\begin{center}
  \includegraphics[width=10cm]{images/stability_PD.eps}
\end{center}
\end{frame}

\begin{frame}\frametitle{Taller}
\begin{enumerate}
  \conti
  \item Considere el sistema representado en variables de estado con
  \begin{align*}
  	A &= 
  	\begin{bmatrix}
  		0  &  1 &  0\\
  		0  &  0 &  1\\
  		-k & -k & -k
  	\end{bmatrix},
  	B = 
  	\begin{bmatrix}
  		0 \\ 0 \\ 1
  	\end{bmatrix}\\
  	C &=
  	\begin{bmatrix}
  		1 & 0 & 0
  	\end{bmatrix}, 
  	D =
  	\begin{bmatrix}
  		0
  	\end{bmatrix}.
  \end{align*}
  \begin{itemize}
  	\item Encuentre la función de transferencia del sistema.
  	\item Para qué valores de $k$ el sistema es estable?
  \end{itemize}
  \seti
\end{enumerate}
\end{frame}

\begin{frame}\frametitle{Taller}
\begin{enumerate}
	\conti
 	\item Considere la ecuación característica:
 	\begin{equation*}
 		s^4 + 2s^3 + (4+K)s^2 + 9s + 25 = 0
 	\end{equation*}
 	Usando el criterio de Routh-Hurwitz, determine el rango de $K$ para estabilidad.
\end{enumerate} 


\end{frame}


\end{document}