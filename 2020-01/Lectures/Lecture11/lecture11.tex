\documentclass[aspectratio=169]{beamer}
% \usepackage[utf8]{inputenc}
\usepackage[T1]{fontenc}
\usetheme{metropolis}
\usecolortheme{orchid}
\usepackage{amsmath}
\usepackage{amssymb}
\usepackage{amsthm}
\usepackage{multirow}
\usepackage[ruled]{algorithm2e}
\usepackage{mathtools}
\usepackage{caption}
% \usepackage{epstopdf}
\usepackage{hyperref}
\usepackage{subcaption}
\usepackage{siunitx}
\usepackage{textcomp}
\usepackage{tcolorbox}
\usepackage{cancel}
\usepackage[framed,numbered]{matlab-prettifier}

\newfontfamily{\lstubuntumono}{Ubuntu Mono}
\newfontfamily{\lstsourcecodepro}{Source Code Pro}
\lstset{
  style              = Matlab-editor,
  basicstyle		 =\lstubuntumono\tiny,
  escapechar         = ",
  mlshowsectionrules = false,
}
 
\usepackage{tikz}
\usetikzlibrary{tikzmark,positioning,arrows.meta,shapes.misc,calc}

\setbeamerfont{footnote}{size=\tiny}

% Information boxes
\newcommand*{\info}[4][16.3]{%
  \node [ annotation, #3, scale=0.65, text width = #1em,
          inner sep = 2mm ] at (#2) {%
  \list{$\bullet$}{\topsep=0pt\itemsep=0pt\parsep=0pt
    \parskip=0pt\labelwidth=8pt\leftmargin=8pt
    \itemindent=0pt\labelsep=2pt}%
    #4
  \endlist
  };
}

%% Definitions for root locus plots
\newcommand*{\rootlocusexample}[4]{% xmin,xmax,ymin,ymax,
		\foreach \x in {#1,...,#2}{
			\ifthenelse{\x=0}{}
		  {% false case
   			\draw (\x cm,1pt) -- (\x cm,-1pt) node[anchor=north] {$\x$};
		  }
		}
		\foreach \y in {#3,...,#4}{
			\ifthenelse{\y=0}{}
		  {% false case
   			\draw (-1pt,\y cm) -- (1pt,\y cm) node[anchor=east] {$\y$};
		  }
   	}
		\draw [-latex] (#1,0) -- (#2,0) node [above]  {$\sigma$};
		\draw [-latex] (0,#3) -- (0,#4) node [right] {$j\omega$};
}

\def\centerarc[#1](#2)(#3:#4:#5)% Syntax: [draw options] (center) (initial angle:final angle:radius)
    { \draw[#1] ($(#2)+({#5*cos(#3)},{#5*sin(#3)})$) arc (#3:#4:#5); }

\tikzset{%
  >={Latex[width=2mm,length=2mm]},
  % Specifications for style of nodes:
            base/.style = {rectangle, rounded corners, draw=black,
                           minimum width=4cm, minimum height=1cm,
                           text centered, font=\sffamily},
  activityStarts/.style = {base, fill=blue!30},
       startstop/.style = {base, fill=red!30},
    activityRuns/.style = {base, fill=green!30},
         process/.style = {base, minimum width=2.5cm, fill=orange!15,
                           font=\ttfamily},
}

\tikzset{pole/.style={cross out, draw=black, minimum size=2*(#1-\pgflinewidth), inner sep=0pt, outer sep=0pt},
pole/.default={3pt}}
\tikzset{poledes/.style={rectangle, draw=black, minimum size=2*(#1-\pgflinewidth), inner sep=0pt, outer sep=0pt},
poledes/.default={3pt}}
\tikzset{zero/.style={circle, draw=black, fill=white, minimum size=2*(#1-\pgflinewidth), inner sep=0pt, outer sep=0pt},
zero/.default={3pt}}
\tikzset{test/.style={rectangle, draw=black, fill=white, minimum size=2*(#1-\pgflinewidth), inner sep=0pt, outer sep=0pt},
test/.default={3pt}}

%% Definitions for block diagrams
\tikzstyle{block} = [draw, fill=blue!20, rectangle, 
    minimum height=2em, minimum width=3em]
\tikzstyle{sum} = [draw, fill=blue!20, circle, node distance=1cm]
\tikzstyle{input} = [coordinate]
\tikzstyle{output} = [coordinate]
\tikzstyle{pinstyle} = [pin edge={to-,thin,black}]

\renewcommand\textbullet{\ensuremath{\bullet}}
\newcommand\scalemath[2]{\scalebox{#1}{\mbox{\ensuremath{\displaystyle #2}}}}
\newcommand{\norm}[1]{\left\lVert#1\right\rVert}

%%% Bibliography
\usepackage[citestyle=numeric,style=numeric,backend=biber,doi=false,isbn=false,url=false]{biblatex}
\addbibresource{references.bib}

%%% Suppress biblatex annoying warning
\usepackage{silence}
\WarningFilter{biblatex}{Patching footnotes failed}

%%% new theorems %%%%%%%%%%%%%%%%%%%%%%%%%%%%%%%%%%%%%%%%%%%%%%%%%%%%%%%%%%%%%%
\theoremstyle{definition}
\newtheorem{mydef}{Definition}

\theoremstyle{plain}
\newtheorem{mylemma}{Lemma}[section]
\newtheorem{mytheorem}{Theorem}[section]
\newtheorem{myproposition}{Proposition}[section]
\newtheorem{myproblem}{Problem}[section]
\newtheorem{mydefinition}{Definition}[section]
\newtheorem{myassumption}{Assumption}[section]

\theoremstyle{remark}
\newtheorem{myremark}{Remark}[section]

\newcounter{saveenumi}
\newcommand{\seti}{\setcounter{saveenumi}{\value{enumi}}}
\newcommand{\conti}{\setcounter{enumi}{\value{saveenumi}}}

\resetcounteronoverlays{saveenumi}

\title{Control de Sistemas}
\subtitle{\small Clase 11: Control y Observación de Sistemas en Variables de Estados}
\author{Gerardo Becerra, Ph.D.}
\institute{Pontificia Universidad Javeriana\\ Departamento de Electrónica}
\date{}

\begin{document}

\frame{\titlepage}	

\begin{frame}[<+->]\frametitle{Agenda}
	\vspace{5mm}
	\tableofcontents
\end{frame}

\begin{frame}[<+->]\frametitle{Introducción}
\centering
\begin{itemize}
	\item Diseño de sistemas de control usando variables de estado.
	\item El sistema es controlado usando la señal de control $u(t)$, que es una función de las variables de estado medibles.
	\item No siempre es posible medir directamente todas las variables de estado de un sistema. $\rightarrow$  Observadores de estados.
	\item Ubicación de polos del sistema de lazo cerrado en lugares donde se satisfacen los requerimientos de desempeño.
	\item Proceso de Diseño:
	\begin{itemize}
		\item Asumiendo que todas las variables de estado son medibles diseñar una ley de control de retroalimentación de estados completos.
		\item Construir un observador de estados para estimar los estados que no son directamente medibles.
		\item Conectar el observador y la retroalimentación de estados al sistema.
	\end{itemize}
\end{itemize}
\end{frame}

\section{Modelos de Sistemas en Variables de Estados}
\begin{frame}[c]\frametitle{Modelos de Sistemas en Variables de Estados}
	Las ecuaciones de cualquier sistema lineal e invariante en el tiempo pueden escribirse de la forma:
	\begin{align*}
		\onslide<2->{\dot{\mathbf{x}}(t) &= \mathbf{Ax}(t) + \mathbf{Bu}(t) \hspace{5mm} \text{(Ec. de Estado)}\\}
		\onslide<3->{\mathbf{y}(t) &= \mathbf{Cx}(t) + \mathbf{Du}(t) \hspace{5mm} \text{(Ec. de Salida)}}
	\end{align*}
	\vspace{-5mm}
	\begin{columns}
		\begin{column}{0.5\textwidth}				
			\begin{itemize}
				\item<4-> $\mathbf{x}(t) \in \mathbb{R}^n$: Variables de estado.
				\item<5-> $\mathbf{u}(t) \in \mathbb{R}^p$: Variables de entrada.
				\item<6-> $\mathbf{y}(t) \in \mathbb{R}^q$: Variables de salida.
			\end{itemize}
		\end{column}
		\begin{column}{0.5\textwidth}	
			\begin{itemize}
				\item<7-> $\mathbf{A} \in \mathbb{R}^{n \times n}$: Matriz del sistema.
				\item<8-> $\mathbf{B} \in \mathbb{R}^{n \times p}$: Matriz de entrada.
				\item<9-> $\mathbf{C} \in \mathbb{R}^{q \times n}$: Matriz de salida.
				\item<10-> $\mathbf{D} \in \mathbb{R}^{q \times p}$: Matriz de transferencia directa.
			\end{itemize}
		\end{column}
	\end{columns}
	\vspace{3mm}		
	\onslide<11->{Por simplicidad, se considerarán sistemas con $p = 1$, $q = 1$.}
\end{frame}

\begin{frame}[c]\frametitle{Sistemas no Lineales}
	Un sistema no lineal puede escribirse de la forma:
	\begin{align*}
		\dot{\mathbf{x}}(t) &= \mathbf{F}(\mathbf{x}(t),\mathbf{u}(t)), \hspace{3mm} F(\cdot,\cdot): \mathbb{R}^n \times \mathbb{R}^p \rightarrow \mathbb{R}^n\\
		\mathbf{y}(t) &= \mathbf{H}(\mathbf{x}(t),\mathbf{u}(t)), \hspace{3mm} H(\cdot,\cdot): \mathbb{R}^n \times \mathbb{R}^p \rightarrow \mathbb{R}^q
	\end{align*}
	\onslide<2->{
	Se definen los puntos de operación (puntos de equilibrio) como:
	\begin{equation*}
		\bar{\mathbf{x}} \in \mathbb{R}^n : 0 = \mathbf{F}(\bar{\mathbf{x}},\bar{\mathbf{u}}),\hspace{2mm} \bar{\mathbf{u}} \in \mathbb{R}^p
	\end{equation*}}
	\onslide<3->{
	Dado un punto de operación $\{\bar{\mathbf{x}}, \bar{\mathbf{u}}\}$, es posible encontrar un sistema lineal e invariante en el tiempo que aproxime al sistema no lineal en la vecindad de $\{\bar{\mathbf{x}}, \bar{\mathbf{u}}\}$} \onslide<4->{$\longrightarrow$ \textcolor{blue}{Linealización}.}
\end{frame}

\begin{frame}[c]\frametitle{Linealización de Sistemas no Lineales}
	Para obtener el modelo linealizado se calcula la matriz Jacobiana (matriz de derivadas parciales) de las funciones $\mathbf{F}$ y $\mathbf{H}$ respecto a las variables $\mathbf{x}$ y $\mathbf{u}$:
	\begin{columns}
		\begin{column}{0.5\textwidth}
			\begin{align*}
				\onslide<2->{	
				A &= \left.\frac{\partial \mathbf{F}}{\partial \mathbf{x}}\right|_{\bar{\mathbf{x}},\bar{\mathbf{u}}} =
				\left.
				\begin{bmatrix}
					\frac{\partial f_1}{\partial x_1} & \frac{\partial f_1}{\partial x_2} & \dots & \frac{\partial f_1}{\partial x_n}\\
					\frac{\partial f_2}{\partial x_1} & \frac{\partial f_2}{\partial x_2} & \dots & \frac{\partial f_2}{\partial x_n}\\
					\vdots & \vdots & \ddots & \vdots\\
					\frac{\partial f_n}{\partial x_1} & \frac{\partial f_n}{\partial x_2} & \dots & \frac{\partial f_n}{\partial x_n}
				\end{bmatrix}
				\right|_{\bar{\mathbf{x}},\bar{\mathbf{u}}}\\}
				\onslide<4->{
				C &= \left.\frac{\partial \mathbf{H}}{\partial \mathbf{x}}\right|_{\bar{\mathbf{x}},\bar{\mathbf{u}}} =
				\left.
				\begin{bmatrix}
					\frac{\partial h_1}{\partial x_1} & \frac{\partial h_1}{\partial x_2} & \dots & \frac{\partial h_1}{\partial x_n}\\
					\frac{\partial h_2}{\partial x_1} & \frac{\partial h_2}{\partial x_2} & \dots & \frac{\partial h_2}{\partial x_n}\\
					\vdots & \vdots & \ddots & \vdots\\
					\frac{\partial h_q}{\partial x_1} & \frac{\partial h_q}{\partial x_2} & \dots & \frac{\partial h_q}{\partial x_n}
				\end{bmatrix}
				\right|_{\bar{\mathbf{x}},\bar{\mathbf{u}}}}
			\end{align*}
		\end{column}
		\begin{column}{0.5\textwidth}
			\begin{align*}
				\onslide<3->{
				B &= \left.\frac{\partial \mathbf{F}}{\partial \mathbf{u}}\right|_{\bar{\mathbf{x}},\bar{\mathbf{u}}} =
				\left.
				\begin{bmatrix}
					\frac{\partial f_1}{\partial u_1} & \frac{\partial f_1}{\partial u_2} & \dots & \frac{\partial f_1}{\partial u_p}\\
					\frac{\partial f_2}{\partial u_1} & \frac{\partial f_2}{\partial u_2} & \dots & \frac{\partial f_2}{\partial u_p}\\
					\vdots & \vdots & \ddots & \vdots\\
					\frac{\partial f_n}{\partial u_1} & \frac{\partial f_n}{\partial u_2} & \dots & \frac{\partial f_n}{\partial u_p}
				\end{bmatrix}
				\right|_{\bar{\mathbf{x}},\bar{\mathbf{u}}}\\}
				\onslide<5->{
				D &= \left.\frac{\partial \mathbf{H}}{\partial \mathbf{u}}\right|_{\bar{\mathbf{x}},\bar{\mathbf{u}}} =
				\left.
				\begin{bmatrix}
					\frac{\partial h_1}{\partial u_1} & \frac{\partial h_1}{\partial u_2} & \dots & \frac{\partial h_1}{\partial u_p}\\
					\frac{\partial h_2}{\partial u_1} & \frac{\partial h_2}{\partial u_2} & \dots & \frac{\partial h_2}{\partial u_p}\\
					\vdots & \vdots & \ddots & \vdots\\
					\frac{\partial h_q}{\partial u_1} & \frac{\partial h_q}{\partial u_2} & \dots & \frac{\partial h_q}{\partial u_p}
				\end{bmatrix}
				\right|_{\bar{\mathbf{x}},\bar{\mathbf{u}}}}
			\end{align*}
		\end{column}
	\end{columns}
\end{frame}

\begin{frame}[c]\frametitle{Linealización de Sistemas no Lineales}
	El sistema linealizado obtenido corresponde a:
	\begin{align*}
		\dot{\hat{\mathbf{x}}}(t) &= \mathbf{A}\hat{\mathbf{x}}(t) + \mathbf{B}\hat{\mathbf{u}}(t)\\
		\hat{\mathbf{y}}(t) &= \mathbf{C}\hat{\mathbf{x}}(t) + \mathbf{D}\hat{\mathbf{u}}(t)
	\end{align*}
	\onslide<2->{
	donde 
	\begin{align*}
		\hat{\mathbf{x}}(t) &= \mathbf{x}(t) - \bar{\mathbf{x}}\\
		\hat{\mathbf{u}}(t) &= \mathbf{u}(t) - \bar{\mathbf{u}}\\
		\hat{\mathbf{y}}(t) &= \mathbf{y}(t) - \bar{\mathbf{y}}
	\end{align*}}
\end{frame}

\begin{frame}[c]\frametitle{Esquema de Control por Retroalimentación de Estados}
	Compensador: Ley de control por retroalimentación de estados + observador de estados.
	\begin{figure}
		\centering
		\includegraphics[width=9cm]{images/stateFeedbackCompensator.eps}
	\end{figure}
\end{frame}

\section{Controlabilidad - Observabilidad}
\begin{frame}[c]\frametitle{Controlabilidad}
	La controlabilidad es necesaria para poder ubicar los polos del sistema en lazo cerrado de forma arbitraria.
	\onslide<2->{
	\begin{definition}[Controlabilidad]
		Un sistema es completamente controlable si existe una señal de control $u(t)$ que logre transferir cualquier estado inicial $x(t_0)$ a cualquier otro estado $x(t)$ en un tiempo finito $t$, con $t_0 \leq t \leq T$.
	\end{definition}}
	\onslide<3->{
	Para verificar la controlabilidad de un sistema, se evalúa si la matriz de controlabilidad es de rango completo:
	\begin{equation*}
		rank\left(\mathcal{P}_c\right) = rank\left(
		\begin{bmatrix}
			\mathbf{B} & \mathbf{AB} & \mathbf{A}^2 \mathbf{B} & \dots & \mathbf{A}^{n-1}\mathbf{B}
		\end{bmatrix}	
		\right) = n
	\end{equation*}}
\end{frame}

\begin{frame}[c]\frametitle{Ejemplo: Controlabilidad de un Sistema}
	Determine si el siguiente sistema es controlable:
	\begin{align*}
		\dot{\mathbf{x}}(t) &=
		\begin{bmatrix}
			-2 & 0\\
			d & -3
		\end{bmatrix}
		\mathbf{x}(t) +
		\begin{bmatrix}
			1\\ 0
		\end{bmatrix}
		u(t)\\
		y(t) &=
		\begin{bmatrix}
			0 & 1
		\end{bmatrix}
		\mathbf{x}(t) +
		\begin{bmatrix}
			0
		\end{bmatrix}
		u(t)
	\end{align*}
	\onslide<2->{
	Construyendo la matriz de controlabilidad:
	\begin{equation*}
		\mathbf{B} =
		\begin{bmatrix}
			1 \\ 0
		\end{bmatrix},
		\hspace{3mm}
		\mathbf{AB} =
		\begin{bmatrix}
			-2 & 0\\
			d & -3
		\end{bmatrix}
		\begin{bmatrix}
			1 \\ 0
		\end{bmatrix} =
		\begin{bmatrix}
			-2\\ d
		\end{bmatrix}
	\end{equation*}}
	\onslide<3->{
	\begin{equation*}
		\mathcal{P}_c =
		\begin{bmatrix}
		 	\mathbf{B} & \mathbf{AB}
		\end{bmatrix} =
		\begin{bmatrix}
			1 & -2\\
			0 & d
		\end{bmatrix}
	\end{equation*}}
	\onslide<4->{
	$d \neq 0 \Longleftrightarrow rank(\mathcal{P}_c) = 2 \Longleftrightarrow$ El sistema es controlable.}
\end{frame}

\begin{frame}[c]\frametitle{Observabilidad}
	La observabilidad se refiere a la posibilidad de estimar las variables de estado.
	\onslide<2->{
	\begin{definition}[Observabilidad]
		Un sistema es completamente observable si y sólo si existe un tiempo finito $T$ tal que el estado inicial $x(t_0)$ puede determinarse a partir de la historia observada $y(t)$ dada una señal de control $u(t)$, para $t_0 \leq t \leq T$.
	\end{definition}}
	\onslide<3->{
	Para verificar la observabilidad de un sistema se evalúa si la matriz de observabilidad es de rango completo:
	\begin{equation*}
		rank(\mathcal{P}_o) = rank\left(
		\begin{bmatrix}
			\mathbf{C} \\ \mathbf{CA} \\ \vdots \\ \mathbf{CA}^{n-1}
		\end{bmatrix}
		\right) = n
	\end{equation*}}
\end{frame}
 
\begin{frame}[c]\frametitle{Ejemplo: Controlabilidad de un Sistema}
	Determine si el siguiente sistema es observable:
	\begin{align*}
		\dot{\mathbf{x}}(t) &=
		\begin{bmatrix}
			-2 & 0\\
			d & -3
		\end{bmatrix}
		\mathbf{x}(t) +
		\begin{bmatrix}
			1\\ 0
		\end{bmatrix}
		u(t)\\
		y(t) &=
		\begin{bmatrix}
			0 & 1
		\end{bmatrix}
		\mathbf{x}(t) +
		\begin{bmatrix}
			0
		\end{bmatrix}
		u(t)
	\end{align*}
	\onslide<2->{
	Construyendo la matriz de observabilidad:
	\begin{equation*}
		\mathbf{C} =
		\begin{bmatrix}
			0 & 1
		\end{bmatrix},
		\hspace{3mm}
		\mathbf{CA} =
		\begin{bmatrix}
			0 & 1
		\end{bmatrix}
		\begin{bmatrix}
			-2 & 0\\
			d & -3
		\end{bmatrix} = 
		\begin{bmatrix}
			d & -3
		\end{bmatrix}
	\end{equation*}}
	\onslide<3->{
	\begin{equation*}
		\mathcal{P}_o = 
		\begin{bmatrix}
			\mathbf{C} \\ \mathbf{CA}
		\end{bmatrix} =
		\begin{bmatrix}
			0 & 1\\
			d & -3
		\end{bmatrix}
	\end{equation*}}
	\onslide<4->{	
	$d \neq 0 \Longleftrightarrow rank(\mathcal{P}_o) = 2 \Longleftrightarrow$ El sistema es observable.}
\end{frame}

\section{Control por Retroalimentación de Estados}
\begin{frame}[c]\frametitle{Control por Retroalimentación de Estados - Problema de Regulación}
	\vspace{2mm}
	\begin{columns}
		\begin{column}{0.3\textwidth}
			\begin{figure}
				\centering
				\includegraphics[width=4cm]{images/fullStateFeedback.eps}
			\end{figure}
			\onslide<2->{
			Modelo en variables de estado: $\dot{\mathbf{x}}(t) = \mathbf{Ax}(t) + \mathbf{B}u(t)$.\\}
			\vspace{3mm}
			\onslide<3->{
			Señal de control: $u(t) = -\mathbf{Kx}(t)$.\\}
		\end{column}
		\begin{column}{0.7\textwidth}
			\onslide<4->{
			Reemplazando $u(t)$ en el modelo:
			\begin{align*}
				\dot{\mathbf{x}}(t) &= \mathbf{Ax}(t) + \mathbf{B}u(t) = \mathbf{Ax}(t) - \mathbf{BKx}(t)\\
				\dot{\mathbf{x}}(t) &= (\mathbf{A} - \mathbf{BK})\mathbf{x}(t)
			\end{align*}}
			\vspace{2mm}
			\onslide<5->{
			Ecuación característica: $\det(\lambda \mathbf{I} - (\mathbf{A} - \mathbf{BK})) = 0$.\\}
			\onslide<6->{
			Si todas las raíces de la ecuación característica se encuentran en el semiplano izquierdo, el sistema de lazo cerrado es estable. Por lo tanto: $\mathbf{x}(t) = e^{(\mathbf{A}-\mathbf{BK})t}\mathbf{x}(t_0) \rightarrow 0$ cuando $t \rightarrow \infty$.\\}
			\vspace{1mm}
			\onslide<7->{	
			Problema de Regulación: Calcular $\mathbf{K}$ tal que la condición inicial $x(t_0) \rightarrow 0$ de acuerdo con los requerimientos de diseño.}
		\end{column}
	\end{columns}
\end{frame}

\begin{frame}[c]\frametitle{Control por Retroalimentación de Estados - Problema de Regulación - Ejemplo}
	\small
	\vspace{3mm}
	\begin{columns}
		\begin{column}{0.5\textwidth}
			Considere el sistema:
			\begin{align*}
				\dot{\mathbf{x}}(t) &=
				\begin{bmatrix}
					0 & 1 & 0\\
					0 & 0 & 1\\
					-2 & -3 & -5
				\end{bmatrix}
				\mathbf{x}(t) +
				\begin{bmatrix}
					0 \\ 0 \\ 1
				\end{bmatrix}u(t)\\
				y(t) &=
				\begin{bmatrix}
					1 & 0 & 0
				\end{bmatrix}
				\mathbf{x}(t)
			\end{align*}
			\onslide<2->{
			Calculando el rango de la matriz de controlabilidad se determina que el sistema es controlable.\\}
			\vspace{3mm}
			\onslide<3->{
			Asumiendo una matriz de retroalimentación de estados $\mathbf{K} = [k_1\ k_2\ k_3]$, el sistema de lazo cerrado es $\dot{\mathbf{x}}(t) = (\mathbf{A}-\mathbf{BK})\mathbf{x}(t)$, donde:}
		\end{column}
		\begin{column}{0.5\textwidth}
			\onslide<3->{
			\begin{equation*}
				\mathbf{A} - \mathbf{BK} = 
				\begin{bmatrix}
					0 & 1 & 0\\
					0 & 0 & 1\\
					-k_1-2 & -k_2-3 & -k_3-5
				\end{bmatrix}
			\end{equation*}}
			\onslide<4->{			
			La ecuación característica es:
			\begin{align*}
				\Delta(\lambda) &= \det(\lambda \mathbf{I} - (\mathbf{A} - \mathbf{BK}))\\
				&= \lambda^3 + (k_3+5)\lambda^2 + (k_2+3)\lambda + (k_1+2) = 0
			\end{align*}}
			\onslide<5->{	
			Se define la ecuación característica deseada como:
			\begin{equation*}
				\Delta_d(\lambda) = (\lambda^2 + 2 \zeta \omega_n \lambda + \omega_n^2)(\lambda + 10\zeta\omega_n)
			\end{equation*}}
		\end{column}
	\end{columns}
\end{frame}

\begin{frame}[c]\frametitle{Control por Retroalimentación de Estados - Problema de Regulación - Ejemplo}
	\small
	\vspace{2mm}
	\begin{columns}
		\begin{column}{0.5\textwidth}	
			Seleccionando $PO = 5\%$, $T_s = 5$ se obtiene $\zeta = 0.69$, $\omega_n = 1.16$. La ecuación característica deseada es:
			\begin{align*}
				\Delta_d(\lambda) &= (\lambda^2 + 0.6\lambda + 36)(\lambda + 4.8)\\
				&= \lambda^3 + 9.6\lambda^2 + 14.14\lambda + 10.75 
			\end{align*}
			\onslide<2->{
			Comparando los polinomios se obtiene:
			\begin{align*}
				k_1 + 2 &= 10.75\\
				k_2 + 3 &= 14.14\\
				k_3 + 5 &= 9.6 
			\end{align*}}
		\end{column}
		\begin{column}{0.5\textwidth}
			\onslide<3->{
			Por lo tanto, la ganancia de retroalimentación de estados es:
			\begin{equation*}
				\mathbf{K} = [8.75,\ 11.14,\ 4.6]
			\end{equation*}}
			\onslide<4->{
			Las raíces del polinomio deseado son $p_{des} = \{-0.8 \pm 0.839, -8\}$. Usando la función \texttt{place} de Matlab:
			\lstinputlisting{code/stateFeedbackPlace.m}}
		\end{column}
	\end{columns}
\end{frame}

\begin{frame}[c]\frametitle{Control por Retroalimentación de Estados - Problema de Regulación - Ejemplo}
Implementación en Simulink del control por retroalimentación de estados.
\begin{figure}
	\centering
	\includegraphics[width=6cm]{images/stateFeedbackRegulatorSchematic.eps}		
\end{figure}	
\end{frame}

\begin{frame}[c]\frametitle{Control por Retroalimentación de Estados - Problema de Regulación - Ejemplo}
	\small
	Respuesta del sistema en lazo cerrado ante condiciones iniciales $x(t_0) = [1,\ 0,\ 0]^\intercal$.
	\vspace{-5mm}
	\begin{columns}
		\begin{column}{0.4\textwidth}
			El estado $x(t)$ converge hacia el punto de equilibrio $\bar{x} = [0,\ 0,\ 0]^\intercal$.\\
			\vspace{3mm}
			La retroalimentación de estados genera la señal de control $u(t)$ necesaria para estabilizar el sistema en el punto de equilibrio dada cualquier condición inicial $x(t_0)$.
		\end{column}
		\begin{column}{0.6\textwidth}
			\begin{figure}
				\centering
				\includegraphics[width=8.5cm]{images/stateFeedbackRegulatorInitial.eps}
			\end{figure}
		\end{column}
	\end{columns}
\end{frame}

\section{Observadores de Estados}
\begin{frame}[<+->]\frametitle{Diseño de Observadores de Estados}
	\begin{itemize}
		\item Durante el diseño del control por retroalimentación de estados se asume que todos los estados del sistema están disponibles para ser retroalimentados.
		\item En la práctica, es posible que sólo algunos de éstos se encuentren disponibles para ser medidos.
		\item En algunos casos, instalar sensores puede incrementar demasiado la complejidad del sistema.
		\item Si el sistema es observable es posible determinar (estimar) los estados que no son directamente medidos (observados).
	\end{itemize}
\end{frame}

\begin{frame}[c]\frametitle{Diseño de Observadores de Estados}
\small
\begin{columns}
	\begin{column}{0.5\textwidth}
		Considere el sistema:
		\begin{align*}
			\dot{\mathbf{x}}(t) &= \mathbf{Ax}(t) + \mathbf{B}u(t)\\
			y(t) &= \mathbf{Cx}(t)
		\end{align*}
		\onslide<2->{
		El observador de estados completos de Luenberger se define como:
		\begin{equation*}
			\dot{\hat{\mathbf{x}}}(t) = \mathbf{A}\hat{\mathbf{x}}(t) + \mathbf{B}u(t) + \mathbf{L}(y(t) - \mathbf{C}\hat{\mathbf{x}}(t))
		\end{equation*}}
		\onslide<3->{
		La matriz $\mathbf{L}$ es la ganancia del observador y se calcula durante el procedimiento de diseño del observador.\\}
	\end{column}
	\begin{column}{0.5\textwidth}
		\onslide<2->{	
		\begin{figure}
			\centering
			\includegraphics[width=6cm]{images/stateObserver.eps}
		\end{figure}}
		\onslide<4->{
		Dos entradas $u(t)$, $y(t)$ y una salida $\hat{\mathbf{x}}(t)$.\\}
		\vspace{3mm}
		\onslide<5->{
		Objetivo: proveer un estimado $\hat{\mathbf{x}}(t)$ tal que  $\hat{\mathbf{x}}(t) \rightarrow \mathbf{x}(t)$ cuando $t \rightarrow \infty$.\\}
		\vspace{3mm}
		\onslide<6->{
		Se define el error de estimación como:
		\begin{equation}
			\mathbf{e}(t) = \mathbf{x}(t) - \hat{\mathbf{x}}(t)
		\end{equation}}
	\end{column}
\end{columns}
\end{frame}

\begin{frame}[c]\frametitle{Diseño de Observadores de Estados}
\small
\vspace{4mm}
\begin{columns}
	\begin{column}{0.5\textwidth}
		Se define el error de estimación como:
		\begin{equation*}
			\mathbf{e}(t) = \mathbf{x}(t) - \hat{\mathbf{x}}(t)
		\end{equation*}
		\onslide<2->{
		Derivando se obtiene:
		\begin{equation*}
			\dot{\mathbf{e}}(t) = \dot{\mathbf{x}}(t) - \dot{\hat{\mathbf{x}}}(t)
		\end{equation*}}
		\onslide<3->{	
		Usando la ecuación de estado y del observador se tiene:
		\begin{align*}
			\dot{\mathbf{e}}(t) &= \mathbf{Ax}(t) + \mathbf{B}u(t) - \mathbf{A}\hat{\mathbf{x}}(t) - \mathbf{B}u(t) - \mathbf{L}(y(t) - \mathbf{C}\hat{\mathbf{x}}(t))\\
			&= \mathbf{A}(\mathbf{x}(t)-\hat{\mathbf{x}}(t)) - \mathbf{L}(y(t) - \mathbf{C}\hat{\mathbf{x}}(t))\\
			&= \mathbf{A}(\mathbf{x}(t)-\hat{\mathbf{x}}(t)) - \mathbf{L}(\mathbf{Cx}(t) - \mathbf{C}\hat{\mathbf{x}}(t))\\
			&= (\mathbf{A} - \mathbf{LC})\mathbf{e}(t)
		\end{align*}}
	\end{column}
	\begin{column}{0.5\textwidth}
		\onslide<4->{
		Se puede garantizar que $\mathbf{e}(t) \rightarrow 0$ cuando $t \rightarrow \infty$ para cualquier error de estimación inicial $\mathbf{e}(t_0)$ si la ecuación característica tiene todas sus raíces en el semiplano izquierdo:
		\begin{equation*}
			\det(\lambda \mathbf{I} - (\mathbf{A} - \mathbf{LC})) = 0
		\end{equation*}}
		\onslide<5->{
		Diseño del observador: encontrar la matriz $\mathbf{L}$ tal que las raíces de la ecuación característica se encuentren en el semiplano izquierdo, asumiendo que el sistema es observable.}
	\end{column}
\end{columns}
\end{frame}

\begin{frame}[c]\frametitle{Diseño de Observadores de Estados - Ejemplo}
	\small
	\vspace{5mm}
	\begin{columns}
		\begin{column}{0.5\textwidth}
			Diseñe un observador de estados para el sistema:
			\begin{align*}
				\dot{\mathbf{x}}(t) &=
				\begin{bmatrix}
					0 & 1 & 0\\
					0 & 0 & 1\\
					-2 & -3 & -5
				\end{bmatrix}
				\mathbf{x}(t) +
				\begin{bmatrix}
					0 \\ 0 \\ 1
				\end{bmatrix}u(t)\\
				y(t) &=
				\begin{bmatrix}
					1 & 0 & 0
				\end{bmatrix}
				\mathbf{x}(t)
			\end{align*}
			\onslide<2->{
			Calculando el rango de la matriz de observabilidad se determina que el sistema es observable.}
		\end{column}
		\begin{column}{0.5\textwidth}
			\onslide<3->{
			Dada la ganancia del observador $\mathbf{L} = [l_1\ l_2\ l_3]^\intercal$, la dinámica del error es $\dot{\mathbf{e}}(t) = (\mathbf{A} - \mathbf{LC})\mathbf{e}(t)$, donde:
			\begin{equation*}
				\mathbf{A} - \mathbf{LC} =
				\begin{bmatrix}
					     -l_1 &  1 &  0\\
					     -l_2 &  0 &  1\\
					- l_3 - 2 & -3 & -5
				\end{bmatrix}
			\end{equation*}}
			\onslide<4->{
			El polinomio característico del sistema es:
			\begin{align*}
				\Delta(\lambda) &= \det(\lambda \mathbf{I} - (\mathbf{A} - \mathbf{LC})) = 0\\
				&= \lambda^3 + (l_1 + 5)\lambda^2 + (5l_1 + l_2 + 3)\lambda\\
				&+ (3l_1 + 5l_2 + l_3 + 2) = 0
			\end{align*}}
		\end{column}
	\end{columns}
\end{frame}

\begin{frame}[c]\frametitle{Diseño de Observadores de Estados - Ejemplo}
	% \small	
	\vspace{3mm}
	\begin{columns}
		\begin{column}{0.5\textwidth}
			Asumiendo $PO = 5\%$, $T_s = 1$ se obtiene $\zeta = 0.69$ y $\omega_n = 5.79$. El polinomio característico deseado es:
			\begin{align*}
				\Delta_d(\lambda) &= (\lambda^2 + 2\zeta\omega_n\lambda + \omega_n^2)(\lambda + 10\zeta\omega_n)\\
				&= \lambda^3 + 48\lambda^2 + 353.6\lambda + 1343.8
			\end{align*}
			\onslide<2->{
			Igualando los coeficientes se obtiene:
			\begin{align*}
				l_1 + 5 &= 48\\
				5l_1 + l_2 + 3 &= 353.6\\
				3l_1 + 5l_2 + l_3 + 2 &= 1343.8
			\end{align*}}
		\end{column}
		\begin{column}{0.5\textwidth}
			\onslide<3->{
			Entonces, la ganancia del observador es:
			\begin{equation*}
				\mathbf{L} = 
				\begin{bmatrix}
					43 & 135.6 & 534.85
				\end{bmatrix}^\intercal
			\end{equation*}}
			\onslide<4->{
			Las raíces del polinomio deseado son $p_{des} = \{-4 \pm 4.1948, -40\}$. Usando la función \texttt{place} de Matlab:
			\lstinputlisting{code/stateObservationPlace.m}}
		\end{column}
	\end{columns}
\end{frame}

\begin{frame}[c]\frametitle{Diseño de Observadores de Estados - Ejemplo}
	Implementación en Simulink del Observador de Estados
	\begin{figure}
		\centering
		\includegraphics[width=8.5cm]{images/stateObservationSchematic.eps}
	\end{figure}
\end{frame}

\begin{frame}[c]\frametitle{Diseño de Observadores de Estados - Ejemplo}
	Respuesta del sistema ante una entrada paso
	\begin{columns}
		\begin{column}{0.3\textwidth}
			El estado estimado converge al estado real en 1 segundo, dada cualquier condición inicial $\mathbf{x}(t_0)$ del sistema.
		\end{column}
		\begin{column}{0.7\textwidth}
			\begin{figure}
				\centering
				\includegraphics[width=9cm]{images/stateObservationStepInitial.eps}
			\end{figure}
		\end{column}
	\end{columns}		
\end{frame}

\section{Control por Retroalimentación y Observación de Estados}
\begin{frame}[<+->]\frametitle{Control por Retroalimentación y Observación de Estados}
	\begin{itemize}
			\item El compensador por variables de estado se construye conectando la ley de control de estado completo al observador de estados.
			\item Luego de diseñar la ley de control y el observador de estados, se puede conectar la salida del observador a la entrada de la retroalimentación de estados:
			\begin{equation}
				u(t) = -\mathbf{K}\hat{\mathbf{x}}(t)
				\label{eq:controlLawObserver}
			\end{equation}
			\item Principio de separación: la ley de control y el observador pueden diseñarse independientemente.
		\end{itemize}	
\end{frame}

\begin{frame}[c]\frametitle{Control por Retroalimentación y Observación de Estados}
	\small
	\vspace{2mm}
	\begin{columns}
		\begin{column}{0.5\textwidth}
			Considere el observador:
			\begin{equation*}
				\dot{\hat{\mathbf{x}}}(t) = \mathbf{A}\hat{\mathbf{x}}(t) + \mathbf{B}u(t) + \mathbf{L}(y(t) - \mathbf{C}\hat{\mathbf{x}}(t))
			\end{equation*}
			\onslide<2->{
			Reemplazando la ley de control \eqref{eq:controlLawObserver} y reorganizando:
			\begin{align*}
				\dot{\hat{\mathbf{x}}}(t) &= (\mathbf{A} - \mathbf{BK} - \mathbf{LC})\hat{\mathbf{x}}(t) + \mathbf{L}y(t)\\
				u(t) &= -\mathbf{K}\hat{\mathbf{x}}(t)
			\end{align*}}
			\onslide<3->{
			Las ecuaciones se pueden representar usando el siguiente diagrama:}
		\end{column}
		\begin{column}{0.5\textwidth}
			\onslide<3->{
			\begin{figure}
				\centering
				\includegraphics[width=7cm]{images/stateObserverSchematic.eps}
			\end{figure}}
			\onslide<4->{
			Reemplazando la ley de control en la ecuación del sistema se tiene:
			\begin{equation*}
				\dot{\mathbf{x}}(t) = \mathbf{Ax}(t) + \mathbf{BK}\hat{\mathbf{x}}(t)
			\end{equation*}}
			\onslide<5->{
			Con $\hat{\mathbf{x}}(t) = \mathbf{x}(t) - \mathbf{e}(t)$ se tiene:
			\begin{equation}
				\dot{\mathbf{x}}(t) = (\mathbf{A} - \mathbf{BK})\mathbf{x}(t) + \mathbf{BK}\mathbf{e}(t)
				\label{eq:stateContObsv}
			\end{equation}}
		\end{column}
	\end{columns}
\end{frame}

\begin{frame}[c]\frametitle{Control por Retroalimentación y Observación de Estados}
	\small
	\vspace{2mm}
	\begin{columns}
		\begin{column}{0.5\textwidth}
			Calculando el error de estimación para el compensador se tiene:
			\begin{align}
				\dot{\mathbf{e}}(t) &= \dot{\mathbf{x}}(t) - \dot{\hat{\mathbf{x}}}(t)\nonumber\\
				&= \mathbf{Ax}(t) + \cancel{\mathbf{B}u(t)} - \mathbf{A}\hat{\mathbf{x}}(t) - \cancel{\mathbf{B}u(t)}\nonumber\\
				&- \mathbf{L}y(t) + \mathbf{LC}\hat{\mathbf{x}}(t)\nonumber\\
				&= \mathbf{A}(\mathbf{x}(t) - \hat{\mathbf{x}}(t)) - \mathbf{LC}(\mathbf{x}(t) - \hat{\mathbf{x}}(t))\nonumber\\
				&= \mathbf{Ae}(t) - \mathbf{LCe}(t)\nonumber\\
				&= (\mathbf{A} - \mathbf{LC})\mathbf{e}(t)
				\label{eq:estimErrorContObsv}
			\end{align}
		\end{column}
		\begin{column}{0.5\textwidth}
			\onslide<2->{
			Reescribiendo las ecuaciones \eqref{eq:stateContObsv}, \eqref{eq:estimErrorContObsv} en forma matricial se tiene:
			\begin{equation}
				\begin{bmatrix}
					\dot{\mathbf{x}}(t) \\ \dot{\mathbf{e}}(t)
				\end{bmatrix} = 
				\begin{bmatrix}
					\mathbf{A} - \mathbf{BK} & \mathbf{BK} \\
					\mathbf{0} & \mathbf{A}-\mathbf{LC}
				\end{bmatrix}
				\begin{bmatrix}
					\mathbf{x}(t)\\ \mathbf{e}(t)
				\end{bmatrix}
				\label{eq:eqCompensator}
			\end{equation}}
			\onslide<3->{
			La ecuación característica asociada al sistema \eqref{eq:eqCompensator} es:
			\begin{equation*}
				\Delta(\lambda) = \det(\lambda \mathbf{I} - (\mathbf{A} - \mathbf{BK}))\det(\lambda \mathbf{I} - (\mathbf{A} - \mathbf{LC}))
			\end{equation*}}
			\onslide<4->{
			Si las raíces de $\det(\lambda \mathbf{I} - (\mathbf{A} - \mathbf{BK})) = 0$ y $\det(\lambda \mathbf{I} - (\mathbf{A} - \mathbf{LC})) = 0$ están en el semiplano izquierdo, todo el sistema es estable.}
		\end{column}
	\end{columns}	
\end{frame}

\begin{frame}[<+->]\frametitle{Control por Retroalimentación y Observación de Estados}
	Procedimiento de Diseño
	\begin{enumerate}
		\item Determinar $\mathbf{K}$ tal que $\det(\lambda \mathbf{I} - (\mathbf{A} - \mathbf{BK})) = 0$ tenga sus raíces en el semiplano izquierdo, en ubicaciones apropiadas de acuerdo con las especificaciones de diseño.
		\item Determinar $\mathbf{L}$ tal que $\det(\lambda \mathbf{I} - (\mathbf{A} - \mathbf{LC})) = 0$ tenga sus raíces en el semiplano izquierdo, en ubicaciones apropiadas para tener un buen desempeño en el observador.
		\item Conectar el observador a la retroalimentación de estados usando:
		\begin{equation*}
			u(t) = - \mathbf{K}\hat{\mathbf{x}}(t)
		\end{equation*}
	\end{enumerate}
\end{frame}

\begin{frame}[c]\frametitle{Control por Retroalimentación y Observación de Estados - Ejemplo}
	Considere el sistema	
	\begin{align*}
		\dot{\mathbf{x}}(t) &=
		\begin{bmatrix}
			0 & 1 & 0\\
			0 & 0 & 1\\
			-2 & -3 & -5
		\end{bmatrix}
		\mathbf{x}(t) +
		\begin{bmatrix}
			0 \\ 0 \\ 1
		\end{bmatrix}u(t)\\
		y(t) &=
		\begin{bmatrix}
			1 & 0 & 0
		\end{bmatrix}
		\mathbf{x}(t)
	\end{align*}
	Ya se realizó el diseño por separado de la retroalimentación de estados y el observador de estados. Por lo tanto, se puede realizar la conexión del observador y la retroalimentación a la planta.
\end{frame}

\begin{frame}[c]\frametitle{Control por Retroalimentación y Observación de Estados - Ejemplo}
	Implementación en Simulink del Compensador (Control y Observador de Estados)
	\begin{figure}
		\centering
		\includegraphics[width=10cm]{images/stateFeedbackRegulatorObservationSchematic.eps}
	\end{figure}
\end{frame}

\begin{frame}[c]\frametitle{Control por Retroalimentación y Observación de Estados - Ejemplo}
	\small
	Respuesta del sistema con compensador ante condiciones iniciales $\mathbf{x}(t_0) = [2,\ 5,\ -1]^\intercal$.
	\vspace{-2mm}
	\begin{columns}
		\begin{column}{0.3\textwidth}
			La velocidad de convergencia de la estimación se ajusta cambiando la posición relativa de los polos del observador respecto a los de la retroalimentación de estados.
		\end{column}
		\begin{column}{0.7\textwidth}
			\begin{figure}
				\centering
				\includegraphics[width=9cm]{images/stateFeedbackRegulatorObservation2.eps}
			\end{figure}
		\end{column}
	\end{columns}
\end{frame}

\section{Seguimiento de Referencias}
\begin{frame}[<+->]\frametitle{Seguimiento de Referencias}
	\begin{itemize}
		\item En el esquema de control por retroalimentación de estados presentado el sistema funciona sólamente como regulador (sin entrada de referencia).
		\item El seguimiento de referencias es muy importante en el diseño de sistemas de control.
		\item Se pueden emplear diferentes métodos para introducir una señal de referencia en el control por retroalimentación de estados.
		\item A continuación se presenta un método que elimina el error de estado estacionario respecto a la referencia y rechaza las perturbaciones.
	\end{itemize}
\end{frame}

\begin{frame}[c]\frametitle{Seguimiento de Referencias}
	\begin{figure}
		\centering
		\includegraphics[width=8cm]{images/stateFeedbackRobust.eps}
	\end{figure}
	\begin{itemize}
		\item<2-> Se introduce una retroalimentación desde la salida que se compara con la referencia.
		\item<3-> El error de la referencia respecto al error se introduce a un integrador.
		\item<4-> El resultado se suma a la salida de la retroalimentación de estados.
	\end{itemize}
\end{frame}

\begin{frame}[c]\frametitle{Seguimiento de Referencias}
	\begin{figure}
		\centering
		\includegraphics[width=8cm]{images/stateFeedbackRobust.eps}
	\end{figure}
	\begin{columns}
		\begin{column}{0.5\textwidth}
			Se agrega una nueva variable de estado:
			\begin{align*}
				\dot{x}_a(t) &= r(t) - y(t) = r(t) - \mathbf{Cx}(t)\\
				u(t) &=
				\begin{bmatrix}
					\mathbf{K} & k_a
				\end{bmatrix}
				\begin{bmatrix}
					\mathbf{x}(t)\\ x_a(t)
				\end{bmatrix}
			\end{align*}
		\end{column}
		\begin{column}{0.5\textwidth}
			\onslide<2->{
			Reemplazando en la ecuación de estado:
			\begin{align*}
				\begin{bmatrix}
					\dot{\mathbf{x}}(t) \\ \dot{x}_a(t)
				\end{bmatrix}
				&=
				\begin{bmatrix}
					\mathbf{A+BK} & \mathbf{B}k_a\\
					-\mathbf{C} & 0
				\end{bmatrix}
				\begin{bmatrix}
					\mathbf{x}(t) \\ x_a(t)
				\end{bmatrix}
				+
				\begin{bmatrix}
					\mathbf{0} \\ 1
				\end{bmatrix}
				r(t)\\
				y(t) &= 
				\begin{bmatrix}
					\textbf{C} & 0
				\end{bmatrix}
				\begin{bmatrix}
					\textbf{x}(t) \\ x_a(t)
				\end{bmatrix}
			\end{align*}}
		\end{column}
	\end{columns}		
\end{frame}

\begin{frame}[c]\frametitle{Seguimiento de Referencias}
	\vspace{3mm}
	\begin{figure}
		\centering
		\includegraphics[width=8cm]{images/statefeedbackrobust.eps}
	\end{figure}
	\begin{columns}
		\begin{column}{0.5\textwidth}
			Para calcular las ganancias $\mathbf{K}$ y $k_a$ se construyen las matrices expandidas
			\begin{equation*}
				\mathbf{A}_e =
				\begin{bmatrix}
					\mathbf{A} & \mathbf{0}\\
					-\mathbf{C} & 0
				\end{bmatrix}, \hspace{3mm}
				\mathbf{B}_e =
				\begin{bmatrix}
					\mathbf{B} \\ 0
				\end{bmatrix}
			\end{equation*}
			\onslide<2->{
			y se utiliza la función \texttt{place} de Matlab:}
		\end{column}
		\begin{column}{0.5\textwidth}
			\onslide<2->{
			\lstinputlisting{code/stateTrackingPlace.m}}
		\end{column}
		\end{columns}		
\end{frame}

\begin{frame}[c]\frametitle{Seguimiento de Referencias - Ejemplo}
	Implementación en Simulink de la retroalimentación de estados con seguimiento de referencia.
	\begin{figure}
		\centering
		\includegraphics[width=10cm]{images/stateFeedbackTrackingSchematic.eps}
	\end{figure}	
\end{frame}

\begin{frame}[c]\frametitle{Seguimiento de Referencias - Ejemplo}
	\small
	Respuesta del sistema con compensador ante condiciones iniciales $\mathbf{x}(t_0) = [2,\ 5,\ -1]^\intercal$.
	\vspace{-2mm}
	\begin{columns}
		\begin{column}{0.3\textwidth}
			La salida del sistema (correspondiente a $x_1(t)$) sigue a la referencia, mientras los otros estados convergen al equilibrio.
		\end{column}
		\begin{column}{0.7\textwidth}
			\begin{figure}
				\centering
				\includegraphics[width=9cm]{images/stateFeedbackTrackingStep.eps}
			\end{figure}
		\end{column}
	\end{columns}
\end{frame}

\section{Compensador con Seguimiento de Referencia}
\begin{frame}[c]\frametitle{Compensador con Seguimiento de Referencia - Ejemplo}
	Implementación en Simulink del compensador (control y observador de estados) con seguimiento de referencia:
	\begin{figure}
		\centering
		\includegraphics[width=11cm]{images/stateFeedbackTrackingObservationSchematic.eps}
	\end{figure}
\end{frame}

\begin{frame}[c]\frametitle{Compensador con Seguimiento de Referencia - Ejemplo}
	\small
	Respuesta del sistema con compensador ante condiciones iniciales $\mathbf{x}(t_0) = [0,\ 0,\ 0]^\intercal$.
	\vspace{-2mm}
	\begin{columns}
		\begin{column}{0.3\textwidth}
			\footnotesize El observador produce un estimado correcto de las variables de estado.\\
			\vspace{3mm}
			El control por retroalimentación de estados garantiza la estabilidad del sistema mientras que la retroalimentación de la salida permite el seguimiento de la referencia.
		\end{column}
		\begin{column}{0.7\textwidth}
			\begin{figure}
				\centering
				\includegraphics[width=9cm]{images/stateFeedbackTrackingObservationStep.eps}
			\end{figure}
		\end{column}
	\end{columns}
\end{frame}

\section{Conclusión}
\begin{frame}[<+->]\frametitle{Conclusión}
	\begin{itemize}
		\item El método de diseño basado en ubicación de polos combinado con observadores es muy útil.
		\item Los polos deseados de lazo cerrado pueden ubicarse de manera arbitraria si la planta es completamente controlable.
		\item Si no todos los estados son medibles, un observador debe ser incorporado para estimarlos.
		\item Durante el diseño, diferentes conjuntos de polos deseados de lazo cerrado pueden considerarse para seleccionar el que genere un mejor desempeño.
	\end{itemize}
\end{frame}

\end{document}